% !TEX root = Dokumentation_SysSpec.tex
\subsection{Use Case Übersicht}
%--------------------------
% Bild Übersicht
%--------------------------
Die folgende Abbildung zeigen die Use Case auf.

\begin{figure}[H]%Position festigen
\centering
\includegraphics[width=0.7\textwidth]{Images/usecase-u.png}
\label{fig:usecase}
\end{figure}


\subsection{Use Case Beschreibung}
In diesem Abschnitt wird die unter Abbildung 2 dargestellte Use Case Übersichtlich tabellarisch genauer beschrieben.

%--------------------------
%Tabellen usecase/1usecase
%--------------------------
\input{usecase/1usecase.tex}
% !TEX root = Dokumentation_SysSpec.tex
\subsubsection{Use Case 2:Bestellung einsehen }
\begin{table}[H]
\begin{tabular}{ | p{0.22\textwidth} | p{0.68\textwidth} |} \hline
\rowcolor{gray!50}
%Titelzeile
	\textbf{Name}          &
	\begin{tabular}{l}
		\textbf{Bestellung einsehen}
	\end{tabular}
	\\ \hline
%Zeile
	\textbf{Kurzbeschreibung}			 &
	\begin{tabular}{l}
		Akteure kann sich alle offenen Bestellungen ansehen.
	\end{tabular}
\\ \hline
%Zeile
	\textbf{Akteure}   		 &
	\begin{tabular}{l}
		Filialleiter, Verkaufspersonal
	\end{tabular}
\\ \hline
%Zeile
	\textbf{Auslöser}              & 
	\begin{tabular}{l}
		Ein Akteure will die Bestellungen einsehen.
	\end{tabular}
\\ \hline
%Zeile
	\textbf{Vorbedingungen}       &
	\begin{tabular}{l}
		-	Der Akteur muss im System angemeldet sein
	
	\end{tabular}		
\\ \hline
%Zeile
	\textbf{Input Information}             &        
	\begin{tabular}{l}
		-	Auswahl aus Liste.

	\end{tabular}
\\ \hline
%Zeile
	\textbf{Ergebnisse}  &
	\begin{tabular}{l}
		Bestellung wird angezeigt.
	\end{tabular}
\\ \hline
%Zeile
	\textbf{Nachbedingung}				 &
	\begin{tabular}{l}
		-
	\end{tabular}
\\ \hline
%Zeile
	\textbf{Ablauf}            &
	\begin{tabular}{l}
		1.	Bestellung aus Liste auswählen. \\
		2.	Doppelklick auf anzuzeigende Bestellung.

	\end{tabular}
\\ \hline
%Zeile
	\textbf{Sonderfälle}			 &
	\begin{tabular}{l}
		-
	\end{tabular}
\\ \hline	
%Untere Abgrenzung
\end{tabular}
\label{tab:2usecase}
\caption{Use Case 2: Bestellung einsehen}
\end{table}
% !TEX root = Dokumentation_SysSpec.tex
\subsubsection{Use Case 3: Bestellung erfassen}
\begin{table}[H]
\begin{tabular}{ | p{0.22\textwidth} | p{0.68\textwidth} |} \hline
\rowcolor{gray!50}
%Titelzeile
	\textbf{Name}          &
	\begin{tabular}{l}
		\textbf{Bestellung erfassen}
	\end{tabular}
	\\ \hline
%Zeile
	\textbf{Kurzbeschreibung}			 &
	\begin{tabular}{l}
		Ein Akteure erfasst aus den bestehenden Produkten eine Bestellungen.
	\end{tabular}
\\ \hline
%Zeile
	\textbf{Akteure}   		 &
	\begin{tabular}{l}
		Filialleiter, Verkaufspersonal
	\end{tabular}
\\ \hline
%Zeile
	\textbf{Auslöser}              & 
	\begin{tabular}{l}
		Kunde möchte via Akteure ein Produkt bestellen.
	\end{tabular}
\\ \hline
%Zeile
	\textbf{Vorbedingungen}       &
	\begin{tabular}{l}
		-	Der Akteur muss im System angemeldet sein (Use Case 1 erfüllt).\\
		-	Der Akteur ist berechtige Bestellungen zu erfassen. \\
		- 	Der Kunde muss bereits im System erfasst sein \\
		- 	Das Produkt muss in der Datenbank (Filliallager) vorhanden sein. 
  

	\end{tabular}		
\\ \hline
%Zeile
	\textbf{Input Information}             &        
	\begin{tabular}{l}
		
		-   Kunde \\
		-	Produkt \\
		-  	Anzahl 
		

	\end{tabular}
\\ \hline
%Zeile
	\textbf{Ergebnisse}  &
	\begin{tabular}{l}
		Verkäufer erfasst von Kunde gewünschte Produkte.
	\end{tabular}
\\ \hline
%Zeile
	\textbf{Nachbedingung}				 &
	\begin{tabular}{l}
		Wenn Bestellung erfasst ist, wird Rechnungswesen benachrichtigt \& \\
		der Kunde erhält eine Bestellbestätigung.
	\end{tabular}
\\ \hline
%Zeile
	\textbf{Ablauf}            &
	\begin{tabular}{l}
		1.	Verkäufer ist am System angemeldet. \\
		2.	Kunde auswählen (Dropdown List). \\
		3.  Produkt in Liste suchen und Anzahl eingeben.\\
		4.  Falls mehrere Produkte gewünscht werden, Schritt 3 wiederholen.\\
		5.	Schaltfläche \grqq Bestellung absenden\grqq{} betätigen. \\
		
	

	\end{tabular}
\\ \hline
%Zeile
	\textbf{Sonderfälle}			 &
	\begin{tabular}{l}
		- Kunde hat offene Mahnung -> Bei der Kundenansicht eine Info.\\
		- Produkt ist unter Lagerbestand -> Nachbestellung auslösen\\
		- Wenn Kunde während Bestellung keine Produkt mehr möchte \\
		  -> Schaltfläche \grqq abbrechen\grqq{}  betätigen. \\
		- Bestellung absenden ohne Produktanzahl eingegeben -> Nullwert Übergabe 
	\end{tabular}
\\ \hline	
%Untere Abgrenzung
\end{tabular}
\label{tab:3usecase}
\caption{Use Case 3: Bestellung erfassen}
\end{table}


\newpage
\textbf{Diagramm: Bestellung erfassen:}

\begin{figure}[H]
\centering
	\includegraphics[width=0.7\linewidth]{Images/Bestellung}
	\caption{Bestellung}
	\label{fig:kontextdiagram}
\end{figure}
\textbf{Beschreibung Diagramm:}
Bestellung erfassen wurde sep. in einem Diagramm aufgezeigt, um eine klar Verständnis zu bekommen. Der erste Schritt ist, dass sich ein Akteur an dem System anmeldet. Hat er eine Berechtigung um Bestellungen auszuführen, kommt er zur Produktewahl. Ansonsten wird ein Popup ausgegeben mit einer Warnung. 
Wenn der Kunde keine Mahnung hat, kann der Akteure das gewünschte Produkt auswählen und mit der Tastatur die gewünschte Anzahle eingeben. Bei mehreren Produkten kann der Vorgang wiederholt werden. Wenn alle gewünschte Produkte ausgewählt sind, kann die Bestellung abgeschlossen werden.


% !TEX root = Dokumentation_SysSpec.tex
\subsubsection{Use Case 4: Bestellung bearbeiten}
\begin{table}[H]
\begin{tabular}{ | p{0.22\textwidth} | p{0.68\textwidth} |} \hline
\rowcolor{gray!50}
%Titelzeile
	\textbf{Name}          &
	\begin{tabular}{l}
		\textbf{Bestellung bearbeiten}
	\end{tabular}
	\\ \hline
%Zeile
	\textbf{Kurzbeschreibung}			 &
	\begin{tabular}{l}
		Ein Akteure kann Bestellungen ändern oder annullieren.
	\end{tabular}
\\ \hline
%Zeile
	\textbf{Akteure}   		 &
	\begin{tabular}{l}
		Filialleiter, Verkaufspersonal
	\end{tabular}
\\ \hline
%Zeile
	\textbf{Auslöser}              & 
	\begin{tabular}{l}
		Eine Bestellung wird bearbeitet.
	\end{tabular}
\\ \hline
%Zeile
	\textbf{Vorbedingungen}       &
	\begin{tabular}{l}
		-	Der Akteur muss im System angemeldet sein (Use Case 1 erfüllt).\\
		-	Der Akteur ist berechtige Bestellungen zu bearbeiten. \\		
		- 	Eine Bestellung muss vorhanden sein. \\
  

	\end{tabular}		
\\ \hline
%Zeile
	\textbf{Input Information}             &        
	\begin{tabular}{l}

		-   Bestellinformationen 
		
		

	\end{tabular}
\\ \hline
%Zeile
	\textbf{Ergebnisse}  &
	\begin{tabular}{l}
		Akteur ändert oder annulliert eine Bestellung.
	\end{tabular}
\\ \hline
%Zeile
	\textbf{Nachbedingung}				 &
	\begin{tabular}{l}
		- Akteur kann die geänderte Bestellung einsehen.\\
		- Rechnungswesen wird benachrichtigt.\\
		- Kunde bekommt neue Bestellbestätigung.
	\end{tabular}
\\ \hline
%Zeile
	\textbf{Ablauf}            &
	\begin{tabular}{l}
		1.	Akteur ist am System angemeldet (Use Case 1 ist erfüllt). \\
		2.	Bestellung aus Liste auswählen. \\
		3.	Schaltfläche \grqq Bestellung annotieren\grqq{} betätigen \\
			um Bestellung zu löschen.\\
		4. 	Schaltfläche \grqq Bestellung editieren\grqq{} betätigen \\
			um Bestellung zu ändern.\\
		5.	Liste mit Bestellungen wird angezeigt \\
		6.	Anzahl der Produkte anpassen\\
		7.  Schaltfläche \grqq Bestellung absenden\grqq{}  betätigen.

	\end{tabular}
\\ \hline
%Zeile
	\textbf{Sonderfälle}			 &
	\begin{tabular}{l}
		- Kunde hat offene Mahnung -> Bei der Kundenansicht eine Info.\\
		- Produkt ist unter Lagerbestand -> Nachbestellung auslösen\\
		- Wenn Kunde während Bestellung keine Produkt mehr möchte \\
		       -> Schaltfläche \grqq abbrechen\grqq{}  betätigen. \\
		- Bestellung abs. ohne Produktanzahl eingeb. -> Nullwert Übergabe 
	\end{tabular}
\\ \hline	
%Untere Abgrenzung
\end{tabular}
\label{tab:4usecase}
\caption{Use Case 4: Bestellung bearbeiten}
\end{table}
% !TEX root = Dokumentation_SysSpec.tex
\subsubsection{Use Case 5: }
\begin{table}[H]
\begin{tabular}{ | p{0.22\textwidth} | p{0.68\textwidth} |} \hline
\rowcolor{gray!50}
%Titelzeile
	\textbf{Name}          &
	\begin{tabular}{l}
		\textbf{Wareineingang erfassen}
	\end{tabular}
	\\ \hline
%Zeile
	\textbf{Kurzbeschreibung}			 &
	\begin{tabular}{l}
		Wenn eine Ware eintrifft wird sie vom Datentypist erfasst.
	\end{tabular}
\\ \hline
%Zeile
	\textbf{Akteure}   		 &
	\begin{tabular}{l}
		Datatypist
	\end{tabular}
\\ \hline
%Zeile
	\textbf{Auslöser}              & 
	\begin{tabular}{l}
		Ware wird geliefert.
	\end{tabular}
\\ \hline
%Zeile
	\textbf{Vorbedingungen}       &
	\begin{tabular}{l}
		-	Der Akteur muss im System angemeldet sein.
		 

	\end{tabular}		
\\ \hline
%Zeile
	\textbf{Input Information}             &        
	\begin{tabular}{l}
		-	Ware wird an Filiale geliefert

	\end{tabular}
\\ \hline
%Zeile
	\textbf{Ergebnisse}  &
	\begin{tabular}{l}
		Filialager wird benachrichtigt.
	\end{tabular}
\\ \hline
%Zeile
	\textbf{Nachbedingung}				 &
	\begin{tabular}{l}
		Ware wird an Kunde ausgeliefert oder abgeholt.
	\end{tabular}
\\ \hline
%Zeile
	\textbf{Ablauf}            &
	\begin{tabular}{l}
		1.	Ware im System erfassen \\
		2.	Kunde benachrichtigen \\
		

	\end{tabular}
\\ \hline
%Zeile
	\textbf{Sonderfälle}			 &
	\begin{tabular}{l}
		- Ware kommt nicht vollständig zur Filiale -> Teillieferung \\oder auf komplett Ware warten. \\
		- Defektes oder unvollständige Lieferung -> Zurück an Filiallager.
	\end{tabular}
\\ \hline	
%Untere Abgrenzung
\end{tabular}
\end{table}
% !TEX root = Dokumentation_SysSpec.tex
\subsubsection{Use Case 6: Logbuch überwachen}
\begin{table}[H]
\begin{tabular}{ | p{0.22\textwidth} | p{0.68\textwidth} |} \hline
\rowcolor{gray!50}
%Titelzeile
	\textbf{Name}          &
	\begin{tabular}{l}
		\textbf{Logbuch überwachen}
	\end{tabular}
	\\ \hline
%Zeile
	\textbf{Kurzbeschreibung}			 &
	\begin{tabular}{l}
		Filialverwalter kann im Logbuch alle Aktivitäten betrachten.
	\end{tabular}
\\ \hline
%Zeile
	\textbf{Akteure}   		 &
	\begin{tabular}{l}
		Filialverwalter
	\end{tabular}
\\ \hline
%Zeile
	\textbf{Auslöser}              & 
	\begin{tabular}{l}
		Filialverwalter will Aktivitäten überprüfen.
	\end{tabular}
\\ \hline
%Zeile
	\textbf{Vorbedingungen}       &
	\begin{tabular}{l}
		-	Der Akteur muss im System angemeldet sein.\\
		-	Im Filial-Bestellsystem sind Aktivitäten abgelaufen .
		

	\end{tabular}		
\\ \hline
%Zeile
	\textbf{Input Information}             &        
	\begin{tabular}{l}
		-	Dokument Log öffnen.

	\end{tabular}
\\ \hline
%Zeile
	\textbf{Ergebnisse}  &
	\begin{tabular}{l}
		Filialverwalter bekommt File mit Aktivitäten.
	\end{tabular}
\\ \hline
%Zeile
	\textbf{Nachbedingung}				 &
	\begin{tabular}{l}
		File kann gespeichert werden.
	\end{tabular}
\\ \hline
%Zeile
	\textbf{Ablauf}            &
	\begin{tabular}{l}
		1.	File öffnen \\
		2.	Log betrachten oder speichern
		

	\end{tabular}
\\ \hline
%Zeile
	\textbf{Sonderfälle}			 &
	\begin{tabular}{l}
		- keine Aktivitäten im System -> File ist leer.\\
		
	\end{tabular}
\\ \hline	
%Untere Abgrenzung
\end{tabular}
\label{tab:6usecase}
\caption{Use Case 6: Logbuch überwachen}
\end{table}
% !TEX root = Dokumentation_SysSpec.tex
\subsubsection{Use Case 7: }
\begin{table}[H]
\begin{tabular}{ | p{0.22\textwidth} | p{0.68\textwidth} |} \hline
\rowcolor{gray!50}
%Titelzeile
	\textbf{Name}          &
	\begin{tabular}{l}
		\textbf{Nachbestellung auslösen}
	\end{tabular}
	\\ \hline
%Zeile
	\textbf{Kurzbeschreibung}			 &
	\begin{tabular}{l}
		Wenn der minimale Lagerbestand eines Produktes erreicht ist, wird automatisch eine Nachbestellung ausgelöst. 
	\end{tabular}
\\ \hline
%Zeile
	\textbf{Akteure}   		 &
	\begin{tabular}{l}
		- 
	\end{tabular}
\\ \hline
%Zeile
	\textbf{Auslöser}              & 
	\begin{tabular}{l}
		Lagerbestand eines Produktes ist unter der minimal  Menge gefallen.
	\end{tabular}
\\ \hline
%Zeile
	\textbf{Vorbedingungen}       &
	\begin{tabular}{l}
		-	Lagerbestände mit minimal Lagerbestände vergleichen. 

	\end{tabular}		
\\ \hline
%Zeile
	\textbf{Input Information}             &        
	\begin{tabular}{l}
		-	Lagerbestand \\
		-	Minimal Lagerbestand

	\end{tabular}
\\ \hline
%Zeile
	\textbf{Ergebnisse}  &
	\begin{tabular}{l}
		Nachbestellung wird ausgelöst.
	\end{tabular}
\\ \hline
%Zeile
	\textbf{Nachbedingung}				 &
	\begin{tabular}{l}
		- 
	\end{tabular}
\\ \hline
%Zeile
	\textbf{Ablauf}            &
	\begin{tabular}{l}
		1.	Lagerbestand von Artikel \\
		2.	Vergleich mit minimal Lagerbestand \\
		3.	Bei Unterschreitung Nachbestellung auslösen.\\
		4.  nächster Vergleich.

	\end{tabular}
\\ \hline
%Zeile
	\textbf{Sonderfälle}			 &
	\begin{tabular}{l}
		- Lagerbestand ist über minimalen Lagerbestand -> keine Nachbestellung auslösen.
	\end{tabular}
\\ \hline	
%Untere Abgrenzung
\end{tabular}
\label{tab:7usecase}
\caption{Use Case 7: Nachbestellung auslösen}
\end{table}
% !TEX root = Dokumentation_SysSpec.tex
\subsubsection{Use Case 8: }
\begin{table}[H]
\begin{tabular}{ | p{0.22\textwidth} | p{0.68\textwidth} |} \hline
\rowcolor{gray!50}
%Titelzeile
	\textbf{Name}          &
	\begin{tabular}{l}
		\textbf{Nachbestellung einsehen}
	\end{tabular}
	\\ \hline
%Zeile
	\textbf{Kurzbeschreibung}			 &
	\begin{tabular}{l}
		Filialleiter kann die Nachbestellungen einsehen.
	\end{tabular}
\\ \hline
%Zeile
	\textbf{Akteure}   		 &
	\begin{tabular}{l}
		Filialleiter
	\end{tabular}
\\ \hline
%Zeile
	\textbf{Auslöser}              & 
	\begin{tabular}{l}
		Filialleiter will Nachbestellungen ansehen.
	\end{tabular}
\\ \hline
%Zeile
	\textbf{Vorbedingungen}       &
	\begin{tabular}{l}
		-	Der Akteur muss im System angemeldet sein (Use Case 1 erfüllt). \\


	\end{tabular}		
\\ \hline
%Zeile
	\textbf{Input Information}             &        
	\begin{tabular}{l}
		-	Liste Nachbestellung Zentrallager
	\end{tabular}
\\ \hline
%Zeile
	\textbf{Ergebnisse}  &
	\begin{tabular}{l}
		Nachbestellungen werden aufgelistet.
	\end{tabular}
\\ \hline
%Zeile
	\textbf{Nachbedingung}				 &
	\begin{tabular}{l}
		-
	\end{tabular}
\\ \hline
%Zeile
	\textbf{Ablauf}            &
	\begin{tabular}{l}
		1.	Schaltfläche \grqq Nachbestellungen\grqq{} betätigen. \\
		2.	Produkt in Liste auswählen. \\
		3.	Schaltfläche \grqq Anzeigen\grqq{} betätigen.

	\end{tabular}
\\ \hline
%Zeile
	\textbf{Sonderfälle}			 &
	\begin{tabular}{l}
		- Es wird eine leere Liste angezeigt wenn keine Nachbestellungen ausgelöst worden sind.
	\end{tabular}
\\ \hline	
%Untere Abgrenzung
\end{tabular}
\label{tab:8usecase}
\caption{Use Case 8: Nachbestellung einsehen}
\end{table}

