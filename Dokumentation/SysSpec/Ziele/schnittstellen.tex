% !TEX root = Dokumentation_SysSpec.tex
\subsection{Externe Schnittstellen}
\subsubsection{Rollen und Akteure}
Im Rahmen des Filial-Bestellsystems existieren folgende Akteure, welche zugleich spezifische Rollen und somit Berechtigungen besitzen.
\begin{itemize}
	\item Benutzer: Jeder Akteur in der Rolle 'Benutzer' kann sich am System mit den entsprechenden Zugangsdaten anmelden.
	\item Filialleiter: besitzt die Rechte 'OrderView' und 'OrderEdit' und kann daher sowohl bestehende Bestellungen einsehen, bearbeiten und annullieren.
	\item Verkaufspersonal: hat grundsätzlich identische Rechte wie Filialleiter, kann jedoch zusätzlich noch Bestellungen erfassen. TODO: Prüfen, ob wirklich so umgesetzt!
	\item Datentypist: besitzt die Rolle 'Supply' und kann dadurch den Wareneingang im System erfassen
	\item Filialverwalter: besitzt die Rolle 'LogView' und kann daher das Logfile mit den abgelaufenen Systemabläufen überprüfen.
	\item Sysadmin: hat keine aktive Steuerungsmöglichkeiten innerhalb des Systems
\end{itemize}
\subsubsection{Datenbankanbindung}
TODO Tobias: OR-Mapper Anbindung beschreiben\\

Für die Persistierung der Daten (ausser Logdaten, RW und Zentrallager) wird eine MySQL-Datenbank aus dem EnterpriseLab mit dem folgenden ConnectionString verwendet:
- TODO ConnectionString
Auf Stufe Datenbank wurden keine weiteren Sichten, Prozeduren und Benutzer eingesetzt. Daher wird gegenüber der Datenbank nur der User grp13 verwendet.
\subsubsection{Zentrallager}
Für das Zentrallager wird eine Stock-Schnittstelle zur Verfügung gestellt. Sie ist in einem Maven-Repo ('https://bintray.com/hslu/maven/appe/5.0.1\#files/ch/hslu/appe/appe\_stock') abgelegt und bereits in das Modul 'appe\_layer\_business' integriert.\\
Die detaillierte Dokumentation zur vorgegebenen Schnittstelle ist unter\\
'https://elearning.hslu.ch/ilias/ilias.php?ref\_id=3290529\&page=APPE\_Startseite\&wpg\_id=11071\&cmd=downloadFile\&cmdClass=ilwikipagegui\&cmdNode=yi:l5:yk\&baseClass=ilwikihandlergui\&file\_id=il\_\_file\_3582976' zu beziehen.

Die Implementation wurde im Business-Layer vorgenommen. Die vorgegebene Schnittstelle wurde nur in einem minimalen Ausmass verwendet. Im Rahmen einer Bestelländerung /-erfassung wird bei Unterschreiten der definierten Mindestanzahl eines Artikels (bis Release 1.0 fix auf 2 definiert) über die Methode 'orderArticle(....)' eine Nachbestellung in der Höhe der geforderten Artikel + 2 ausgelöst. Es wird dabei nicht geprüft, ob am Zentrallager noch entsprechende Artikel vorhanden sind. Es wird daher davon ausgegangen, dass am Zentrallager wie auch im Filiallager ein Negativlagerbestand möglich ist.\\
TODO Severin: Klassendiagram hinzufügen
\subsubsection{Rechnungswesen}
Das Rechnungswesen hat die Aufgabe bei Erfassung / Änderung einer Bestellung eine entsprechende Rechnung an den Kunden zu versenden.\\
Von seiten Auftraggeber sind keine Schnittstellen zur Verfügung gestellt worden und daher wurden die entsprechenden Aufgaben des Rechnungswesen als simpler Stub implementiert, der lediglich ein Output über die ausgeführte Arbeit (z.B. Neue Rechnung gedruckt) protokolliert.\\
TODO Severin: Klassendiagram hinzufügen

\subsection{Interne Schnittstellen}
In diesem Kapitel werden nur wichtige, systemrelevante Schnittstellen spezifiziert. Es werden primär die Schnittstellen zwischen den Layern 'Client', 'Business' inkl. Remote und 'Data' beschrieben.

\subsubsection{Globale Sicht}
In einer groben Übersicht wird die globale Sicht der Schnittstellen zwischen den Schichten / Packages aufgezeigt.:\\
TODO : BigPicture Schnittstellen

\subsubsection{Model}
TODO Severin
\subsubsection{Client <-> Business-Remote}
TODO Ramon
\subsubsection{Business-Remote <-> Business-Core}
TODO Marco
\subsubsection{Business-Core <-> Data}
TODO Marco \&Tobias
