% !TEX root = Dokumentation_SysSpec.tex
\subsection{Robustheit}
\begin{itemize}
	\item Ausfälle von Umsystemen abfangen.
	\item Datentyp-Mismatch zwischen Applikation und Datenbank abfangen (Overflows durch unterschiedliche Längendefinition)
\end{itemize}

\subsection{Performance \& Skalierung}
\begin{itemize}
	\item Datenbank-Notifications, wenn Datenbank erweitert werden muss (bspw durch max. Anzahl möglicher Bestellungen)
\end{itemize}

\subsection{Sicherheit}
\begin{itemize}
	\item Einsatz von 2-Faktor-Authentifizierung durch EInbindung von Open-Source-OTP-Software, bspw. Google Authenticator oder Microsoft Authenticator
	\item Verschlüsselung der Applikationsdaten (Stufe Datenbank) und der übertragenene Daten
	\item Applikation, Datenbank und Periphere Systeme prüfen die Identität der Umsysteme, bspw über Zertifikate.
	\item Applikation darf nur "lokale" Aufrufe ausführen. Sicherstellung, dass Applikation nicht über das Internet, sondern üer Unternehmens-Netzwerke kommuniziert (VPN-Tunnel, Trusted Networks)
	\item Usersessions mit dediziertem Sessionmanagement (bspw. Cookies) lösen
\end{itemize}

\subsection{Erweiterungen \& Features}
\begin{itemize}
	\item Anbindung einer zentralen Benutzerverwaltung (Applikationsbenutzer) wie AD
	\item Anbindung einer zentralen Kundenverwaltung zum Vermeiden von Redundanzen mit Rechnungswesen. Führt zu klarer Trennung von Aufgabenbereichen der Applikation.
	\item Einbindung einer zentralen Log-Infrastruktur wie Syslog
	\item Einbindung von kritischen Events für SIEM-Systeme
\end{itemize}