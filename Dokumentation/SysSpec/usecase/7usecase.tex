% !TEX root = Dokumentation_SysSpec.tex
\subsubsection{Use Case 7: Nachbestellung auslösen}
\begin{table}[H]
\begin{tabular}{ | p{0.22\textwidth} | p{0.68\textwidth} |} \hline
\rowcolor{gray!50}
%Titelzeile
	\textbf{Name}          &
	\begin{tabular}{l}
		\textbf{Nachbestellung auslösen}
	\end{tabular}
	\\ \hline
%Zeile
	\textbf{Kurzbeschreibung}			 &
	\begin{tabular}{l}
		Der Lagerbestand wird mit dem definierten Grenzwert verglichen und eine Bestellung ausgelöst.
	\end{tabular}
\\ \hline
%Zeile
	\textbf{Akteure}   		 &
	\begin{tabular}{l}
		- 
	\end{tabular}
\\ \hline
%Zeile
	\textbf{Auslöser}              & 
	\begin{tabular}{l}
		Lagerbestand ist unter den definierten Grenzwert gefallen.
	\end{tabular}
\\ \hline
%Zeile
	\textbf{Vorbedingungen}       &
	\begin{tabular}{l}
		-	Bestellung wird abgeschlossen.

	\end{tabular}		
\\ \hline
%Zeile
	\textbf{Input Information}             &        
	\begin{tabular}{l}
		-	Lagerbestand. \\
		-   Bestand Nachbestellung. \\
		-	Definierter Grenzwert.\\

	\end{tabular}
\\ \hline
%Zeile
	\textbf{Ergebnisse}  &
	\begin{tabular}{l}
		Nachbestellung wird ausgelöst.
	\end{tabular}
\\ \hline
%Zeile
	\textbf{Nachbedingung}				 &
	\begin{tabular}{l}
		- 
	\end{tabular}
\\ \hline
%Zeile
	\textbf{Ablauf}            &
	\begin{tabular}{l}
		1.  Bestellung wird ausgelöst.\
		2.	Lagerbestand von Artikel. \\
		3.	Vergleich mit definierten Grenzwert. \\
		4.	Prüfen ob Nachbestellung aktiv.

	\end{tabular}
\\ \hline
%Zeile
	\textbf{Sonderfälle}			 &
	\begin{tabular}{l}
		- Lagerbestand ist über minimalen Lagerbestand -> keine Nachbestellung auslösen.
	\end{tabular}
\\ \hline	
%Untere Abgrenzung
\end{tabular}
\label{tab:7usecase}
\caption{Use Case 7: Nachbestellung auslösen}
\end{table}