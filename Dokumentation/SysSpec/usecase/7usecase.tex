% !TEX root = Dokumentation_SysSpec.tex
\subsubsection{Use Case 7: }
\begin{table}[H]
\begin{tabular}{ | p{0.22\textwidth} | p{0.68\textwidth} |} \hline
\rowcolor{gray!50}
%Titelzeile
	\textbf{Name}          &
	\begin{tabular}{l}
		\textbf{Nachbestellung auslösen}
	\end{tabular}
	\\ \hline
%Zeile
	\textbf{Kurzbeschreibung}			 &
	\begin{tabular}{l}
		Wenn der minimale Lagerbestand eines Produktes erreicht ist, wird automatisch eine Nachbestellung ausgelöst. 
	\end{tabular}
\\ \hline
%Zeile
	\textbf{Akteure}   		 &
	\begin{tabular}{l}
		- 
	\end{tabular}
\\ \hline
%Zeile
	\textbf{Auslöser}              & 
	\begin{tabular}{l}
		Lagerbestand eines Produktes ist unter der minimal  Menge gefallen.
	\end{tabular}
\\ \hline
%Zeile
	\textbf{Vorbedingungen}       &
	\begin{tabular}{l}
		-	Lagerbestände mit minimal Lagerbestände vergleichen. 

	\end{tabular}		
\\ \hline
%Zeile
	\textbf{Input Information}             &        
	\begin{tabular}{l}
		-	Lagerbestand \\
		-	Minimal Lagerbestand

	\end{tabular}
\\ \hline
%Zeile
	\textbf{Ergebnisse}  &
	\begin{tabular}{l}
		Nachbestellung wird ausgelöst.
	\end{tabular}
\\ \hline
%Zeile
	\textbf{Nachbedingung}				 &
	\begin{tabular}{l}
		- 
	\end{tabular}
\\ \hline
%Zeile
	\textbf{Ablauf}            &
	\begin{tabular}{l}
		1.	Lagerbestand von Artikel \\
		2.	Vergleich mit minimal Lagerbestand \\
		3.	Bei Unterschreitung Nachbestellung auslösen.\\
		4.  nächster Vergleich.

	\end{tabular}
\\ \hline
%Zeile
	\textbf{Sonderfälle}			 &
	\begin{tabular}{l}
		- Lagerbestand ist über minimalen Lagerbestand -> keine Nachbestellung auslösen.
	\end{tabular}
\\ \hline	
%Untere Abgrenzung
\end{tabular}
\label{tab:7usecase}
\caption{Use Case 7: Nachbestellung auslösen}
\end{table}