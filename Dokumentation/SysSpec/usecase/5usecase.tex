% !TEX root = Dokumentation_SysSpec.tex
\subsubsection{Use Case 5: Wareneingang erfassen}
\begin{table}[H]
\begin{tabular}{ | p{0.22\textwidth} | p{0.68\textwidth} |} \hline
\rowcolor{gray!50}
%Titelzeile
	\textbf{Name}          &
	\begin{tabular}{l}
		\textbf{Wareneingang erfassen}
	\end{tabular}
	\\ \hline
%Zeile
	\textbf{Kurzbeschreibung}			 &
	\begin{tabular}{l}
		Wenn eine Ware eintrifft wird sie vom Datentypist erfasst.
	\end{tabular}
\\ \hline
%Zeile
	\textbf{Akteure}   		 &
	\begin{tabular}{l}
		Datentypist
	\end{tabular}
\\ \hline
%Zeile
	\textbf{Auslöser}              & 
	\begin{tabular}{l}
		Ware wird geliefert.
	\end{tabular}
\\ \hline
%Zeile
	\textbf{Vorbedingungen}       &
	\begin{tabular}{l}
		-	Der Akteur muss im System angemeldet sein (Use Case 1 erfüllt).
		 

	\end{tabular}		
\\ \hline
%Zeile
	\textbf{Input Information}             &        
	\begin{tabular}{l}
		-	Anzahl Produkte.

	\end{tabular}
\\ \hline
%Zeile
	\textbf{Ergebnisse}  &
	\begin{tabular}{l}
		Anzahl Ware wird im Filiallager aktualisiert. 
	\end{tabular}
\\ \hline
%Zeile
	\textbf{Nachbedingung}				 &
	\begin{tabular}{l}
		- 
	\end{tabular}
\\ \hline
%Zeile
	\textbf{Ablauf}            &
	\begin{tabular}{l}
		1.	gelieferte Ware überprüfen \\
		2.	Lagerbestand im Filiallager aktualisieren.
		

	\end{tabular}
\\ \hline
%Zeile
	\textbf{Sonderfälle}			 &
	\begin{tabular}{l}
		- Ware kommt nicht vollständig zur Filiale -> Teillieferung \\oder auf komplett Ware warten. \\
		- Defektes oder unvollständige Lieferung -> Zurück an Zentrallager.
	\end{tabular}
\\ \hline	
%Untere Abgrenzung
\end{tabular}
\label{tab:5usecase}
\caption{Use Case 5: Wareneingang erfassen}
\end{table}