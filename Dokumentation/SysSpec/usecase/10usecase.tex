% !TEX root = Dokumentation_SysSpec.tex
\subsubsection{Use Case 10: }
\begin{table}[H]
\begin{tabular}{ | p{0.22\textwidth} | p{0.68\textwidth} |} \hline
\rowcolor{gray!50}
%Titelzeile
	\textbf{Name}          &
	\begin{tabular}{l}
		\textbf{Artikel bearbeiten}
	\end{tabular}
	\\ \hline
%Zeile
	\textbf{Kurzbeschreibung}			 &
	\begin{tabular}{l}
		Ein Akteur erfasst oder ändert ein Produkt. 
	\end{tabular}
\\ \hline
%Zeile
	\textbf{Akteure}   		 &
	\begin{tabular}{l}
		Filialleiter
	\end{tabular}
\\ \hline
%Zeile
	\textbf{Auslöser}              & 
	\begin{tabular}{l}
		- Akteur will ein neues Produkt im System erfassen.\\
		- Akteur will ein bestehendes Produkt anpassen.
	\end{tabular}
\\ \hline
%Zeile
	\textbf{Vorbedingungen}       &
	\begin{tabular}{l}
		-	Der Akteur muss im System angemeldet sein (Use Case 1 erfüllt). \\


	\end{tabular}		
\\ \hline
%Zeile
	\textbf{Input Information}             &        
	\begin{tabular}{l}
		-	Produktname \\
		- 	Preis \\
		- 	Lagerbestand
	\end{tabular}
\\ \hline
%Zeile
	\textbf{Ergebnisse}  &
	\begin{tabular}{l}
		- Neues Produkt ist im System erfasst.\\
		- Produkt wird geändert an mindestens einem Attribut.
	\end{tabular}
\\ \hline
%Zeile
	\textbf{Nachbedingung}				 &
	\begin{tabular}{l}
		- Lagerbestand wird überprüft und evtl. Nachbestellung ausgelöst
	\end{tabular}
\\ \hline
%Zeile
	\textbf{Ablauf}            &
	\begin{tabular}{l}
		1.	Schaltfläche \grqq Artikel\grqq{} betätigen. \\
		2.	Schaltfläche \grqq Neu\grqq{} oder \grqq Ändern\grqq{} betätigen. \\
		3.	Produktname eingeben. \\
		4.  Preis eingeben. \\
		5.  Lagerbestand eingeben. \\
		6.  Schaltfläche \grqq Speichern\grqq{} betätigen.

	\end{tabular}
\\ \hline
%Zeile
	\textbf{Sonderfälle}			 &
	\begin{tabular}{l}
		-
	\end{tabular}
\\ \hline	
%Untere Abgrenzung
\end{tabular}
\label{tab:10usecase}
\caption{Use Case 10: Artikel bearbeiten}
\end{table}