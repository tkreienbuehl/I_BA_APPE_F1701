% !TEX root = Dokumentation_SysSpec.tex
\subsubsection{Use Case 3: Bestellung erfassen}
\begin{table}[H]
\begin{tabular}{ | p{0.22\textwidth} | p{0.68\textwidth} |} \hline
\rowcolor{gray!50}
%Titelzeile
	\textbf{Name}          &
	\begin{tabular}{l}
		\textbf{Bestellung erfassen}
	\end{tabular}
	\\ \hline
%Zeile
	\textbf{Kurzbeschreibung}			 &
	\begin{tabular}{l}
		Ein Akteure erfasst aus den bestehenden Produkten eine Bestellungen.
	\end{tabular}
\\ \hline
%Zeile
	\textbf{Akteure}   		 &
	\begin{tabular}{l}
		Filialleiter, Verkaufspersonal
	\end{tabular}
\\ \hline
%Zeile
	\textbf{Auslöser}              & 
	\begin{tabular}{l}
		Kunde möchte via Akteure ein Produkt bestellen.
	\end{tabular}
\\ \hline
%Zeile
	\textbf{Vorbedingungen}       &
	\begin{tabular}{l}
		-	Der Akteur muss im System angemeldet sein (Use Case 1 erfüllt).\\
		-	Der Akteur ist berechtige Bestellungen zu erfassen. \\
		- 	Der Kunde muss bereits im System erfasst sein \\
		- 	Das Produkt muss in der Datenbank (Filliallager) vorhanden sein. 
  

	\end{tabular}		
\\ \hline
%Zeile
	\textbf{Input Information}             &        
	\begin{tabular}{l}
		
		-   Kunde \\
		-	Produkt \\
		-  	Anzahl 
		

	\end{tabular}
\\ \hline
%Zeile
	\textbf{Ergebnisse}  &
	\begin{tabular}{l}
		Verkäufer erfasst von Kunde gewünschte Produkte.
	\end{tabular}
\\ \hline
%Zeile
	\textbf{Nachbedingung}				 &
	\begin{tabular}{l}
		Wenn Bestellung erfasst ist, wird Rechnungswesen benachrichtigt \& \\
		der Kunde erhält eine Bestellbestätigung.
	\end{tabular}
\\ \hline
%Zeile
	\textbf{Ablauf}            &
	\begin{tabular}{l}
		1.	Verkäufer ist am System angemeldet. \\
		2.	Kunde auswählen (Dropdown List). \\
		3.  Produkt in Liste suchen und Anzahl eingeben.\\
		4.  Falls mehrere Produkte gewünscht werden, Schritt 3 wiederholen.\\
		5.	Schaltfläche \grqq Bestellung absenden\grqq{} betätigen. \\
		
	

	\end{tabular}
\\ \hline
%Zeile
	\textbf{Sonderfälle}			 &
	\begin{tabular}{l}
		- Kunde hat offene Mahnung -> Bei der Kundenansicht eine Info.\\
		- Produkt ist unter Lagerbestand -> Nachbestellung auslösen\\
		- Wenn Kunde während Bestellung keine Produkt mehr möchte \\
		  -> Schaltfläche \grqq abbrechen\grqq{}  betätigen. \\
		- Bestellung abs. ohne Produktanzahl eing. -> Nullwert Übergabe 
	\end{tabular}
\\ \hline	
%Untere Abgrenzung
\end{tabular}
\label{tab:3usecase}
\caption{Use Case 3: Bestellung erfassen}
\end{table}