% !TEX root = Dokumentation_SysSpec.tex
\subsubsection{Use Case 4: }
\begin{table}[H]
\begin{tabular}{ | p{0.22\textwidth} | p{0.68\textwidth} |} \hline
\rowcolor{gray!50}
%Titelzeile
	\textbf{Name}          &
	\begin{tabular}{l}
		\textbf{Bestellung bearbeiten}
	\end{tabular}
	\\ \hline
%Zeile
	\textbf{Kurzbeschreibung}			 &
	\begin{tabular}{l}
		Ein Akteure kann Bestellungen ändern oder annullieren
	\end{tabular}
\\ \hline
%Zeile
	\textbf{Akteure}   		 &
	\begin{tabular}{l}
		Filialleiter, Verkaufspersonal
	\end{tabular}
\\ \hline
%Zeile
	\textbf{Auslöser}              & 
	\begin{tabular}{l}
		Eine Bestellung wird bearbeitet.
	\end{tabular}
\\ \hline
%Zeile
	\textbf{Vorbedingungen}       &
	\begin{tabular}{l}
		-	Der Akteur muss im System angemeldet sein (Use Case 1 erfühlt).\\
		-	Der Akteur ist berechtige Bestellungen zu bearbeiten. \\
		
		- 	Die Bestellung muss vorhanden sein. \\
  

	\end{tabular}		
\\ \hline
%Zeile
	\textbf{Input Information}             &        
	\begin{tabular}{l}
		- 	Verkäufer \\
		-	Kunde \\
		-   Bestellung des Kunden \\
		-	Produkt \\
		-  	Anzahl 
		

	\end{tabular}
\\ \hline
%Zeile
	\textbf{Ergebnisse}  &
	\begin{tabular}{l}
		Verkäufer ändert oder annulliert eine Bestellung.
	\end{tabular}
\\ \hline
%Zeile
	\textbf{Nachbedingung}				 &
	\begin{tabular}{l}
		- Verkäufer kann die geänderte Bestellung einsehen.\\
		- Rechnungswesen wird benachrichtigt.\\
		- Kunde bekommt neue Bestellbestätigung.
	\end{tabular}
\\ \hline
%Zeile
	\textbf{Ablauf}            &
	\begin{tabular}{l}
		1.	Verkäufer ist am System angemeldet. \\
		2.	Schaltfläche \grqq Bestellungen bearbeiten\grqq{} auswählen.
		3.	Bestellung aus Liste auswählen. \\
		4. 	Bestellung bearbeiten.\\
		5.	Schaltfläche \grqq Hinzufügen\grqq{}  oder \grqq Löschen\grqq{} betätigen. \\
		6.	Falls mehrere Bestellungen bearbeitet werden möchten\\
		 Schritt 3 bis 5 wiederholen.\\
		8.  Schaltfläche \grqq Bestellung abschliessen\grqq{}  betätigen.

	\end{tabular}
\\ \hline
%Zeile
	\textbf{Sonderfälle}			 &
	\begin{tabular}{l}
		- Kunde hat offene Mahnung -> Bei der Kundenansicht eine Info.\\
		- Produkt ist nicht im Lager -> Nachbestellung auslösen
	\end{tabular}
\\ \hline	
%Untere Abgrenzung
\end{tabular}
\label{tab:4usecase}
\caption{Use Case 4: Bestellung bearbeiten}
\end{table}