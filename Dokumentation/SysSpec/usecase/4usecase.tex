% !TEX root = Dokumentation_SysSpec.tex
\subsubsection{Use Case 4: }
\begin{table}[H]
\begin{tabular}{ | p{0.22\textwidth} | p{0.68\textwidth} |} \hline
\rowcolor{gray!50}
%Titelzeile
	\textbf{Name}          &
	\begin{tabular}{l}
		\textbf{Anmeldung}
	\end{tabular}
	\\ \hline
%Zeile
	\textbf{Kurzbeschreibung}			 &
	\begin{tabular}{l}
		Ein Benutzer meldet sich im System an
	\end{tabular}
\\ \hline
%Zeile
	\textbf{Akteure}   		 &
	\begin{tabular}{l}
		Filialleiter, Verkaufspersonal, Sysadmin, Filialverwalter, Datatypist
	\end{tabular}
\\ \hline
%Zeile
	\textbf{Auslöser}              & 
	\begin{tabular}{l}
		Der Akteure will sich am System anmelden
	\end{tabular}
\\ \hline
%Zeile
	\textbf{Vorbedingungen}       &
	\begin{tabular}{l}
		-	Der Akteur muss im System vorhanden sein \\
		-	Benutzername muss vorhanden sein \\
		-	Passwort muss vorhanden sein 

	\end{tabular}		
\\ \hline
%Zeile
	\textbf{Input Information}             &        
	\begin{tabular}{l}
		-	Username \\
		-	Passwort

	\end{tabular}
\\ \hline
%Zeile
	\textbf{Ergebnisse}  &
	\begin{tabular}{l}
		Benutzer ist am System angemeldet.
	\end{tabular}
\\ \hline
%Zeile
	\textbf{Nachbedingung}				 &
	\begin{tabular}{l}
		Benutzer kann im System je nach definierter Rolle arbeiten.
	\end{tabular}
\\ \hline
%Zeile
	\textbf{Ablauf}            &
	\begin{tabular}{l}
		1.	Applikation starten \\
		2.	Username \& Passwort eingeben \\
		3.	Schaltfläche Login betätigen

	\end{tabular}
\\ \hline
%Zeile
	\textbf{Sonderfälle}			 &
	\begin{tabular}{l}
		Passwort wird 3-mal falsch eingegeben.
	\end{tabular}
\\ \hline	
%Untere Abgrenzung
\end{tabular}
\end{table}