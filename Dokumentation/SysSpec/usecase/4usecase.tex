% !TEX root = Dokumentation_SysSpec.tex
\subsubsection{Use Case 4: Bestellung bearbeiten}
\begin{table}[H]
\begin{tabular}{ | p{0.22\textwidth} | p{0.68\textwidth} |} \hline
\rowcolor{gray!50}
%Titelzeile
	\textbf{Name}          &
	\begin{tabular}{l}
		\textbf{Bestellung bearbeiten}
	\end{tabular}
	\\ \hline
%Zeile
	\textbf{Kurzbeschreibung}			 &
	\begin{tabular}{l}
		Ein Akteure kann Bestellungen ändern oder annullieren.
	\end{tabular}
\\ \hline
%Zeile
	\textbf{Akteure}   		 &
	\begin{tabular}{l}
		Filialleiter, Verkaufspersonal
	\end{tabular}
\\ \hline
%Zeile
	\textbf{Auslöser}              & 
	\begin{tabular}{l}
		Eine Bestellung wird bearbeitet.
	\end{tabular}
\\ \hline
%Zeile
	\textbf{Vorbedingungen}       &
	\begin{tabular}{l}
		-	Der Akteur muss im System angemeldet sein (Use Case 1 erfüllt).\\
		-	Der Akteur ist berechtige Bestellungen zu bearbeiten. \\		
		- 	Eine Bestellung muss vorhanden sein. \\
  

	\end{tabular}		
\\ \hline
%Zeile
	\textbf{Input Information}             &        
	\begin{tabular}{l}

		-   Bestellinformationen 
		
		

	\end{tabular}
\\ \hline
%Zeile
	\textbf{Ergebnisse}  &
	\begin{tabular}{l}
		Akteur ändert oder annulliert eine Bestellung.
	\end{tabular}
\\ \hline
%Zeile
	\textbf{Nachbedingung}				 &
	\begin{tabular}{l}
		- Akteur kann die geänderte Bestellung einsehen.\\
		- Rechnungswesen wird benachrichtigt.\\
		- Kunde bekommt neue Bestellbestätigung.
	\end{tabular}
\\ \hline
%Zeile
	\textbf{Ablauf}            &
	\begin{tabular}{l}
		1.	Akteur ist am System angemeldet (Use Case 1 ist erfüllt). \\
		2.	Bestellung aus Liste auswählen. \\
		3.	Schaltfläche \grqq Bestellung annotieren\grqq{} betätigen \\
			um Bestellung zu löschen.\\
		4. 	Schaltfläche \grqq Bestellung editieren\grqq{} betätigen \\
			um Bestellung zu ändern.\\
		5.	Liste mit Bestellungen wird angezeigt \\
		6.	Anzahl der Produkte anpassen\\
		7.  Schaltfläche \grqq Bestellung absenden\grqq{}  betätigen.

	\end{tabular}
\\ \hline
%Zeile
	\textbf{Sonderfälle}			 &
	\begin{tabular}{l}
		- Kunde hat offene Mahnung -> Bei der Kundenansicht eine Info.\\
		- Produkt ist unter Lagerbestand -> Nachbestellung auslösen\\
		- Wenn Kunde während Bestellung keine Produkt mehr möchte \\
		       -> Schaltfläche \grqq abbrechen\grqq{}  betätigen. \\
		- Bestellung abs. ohne Produktanzahl eingeb. -> Nullwert Übergabe 
	\end{tabular}
\\ \hline	
%Untere Abgrenzung
\end{tabular}
\label{tab:4usecase}
\caption{Use Case 4: Bestellung bearbeiten}
\end{table}