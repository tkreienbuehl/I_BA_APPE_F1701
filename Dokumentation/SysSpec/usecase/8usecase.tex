% !TEX root = Dokumentation_SysSpec.tex
\subsubsection{Use Case 8: }
\begin{table}[H]
\begin{tabular}{ | p{0.22\textwidth} | p{0.68\textwidth} |} \hline
\rowcolor{gray!50}
%Titelzeile
	\textbf{Name}          &
	\begin{tabular}{l}
		\textbf{Nachbestellung einsehen}
	\end{tabular}
	\\ \hline
%Zeile
	\textbf{Kurzbeschreibung}			 &
	\begin{tabular}{l}
		Filialleiter kann die Nachbestellungen einsehen.
	\end{tabular}
\\ \hline
%Zeile
	\textbf{Akteure}   		 &
	\begin{tabular}{l}
		Filialleiter
	\end{tabular}
\\ \hline
%Zeile
	\textbf{Auslöser}              & 
	\begin{tabular}{l}
		Filialleiter will Nachbestellungen ansehen.
	\end{tabular}
\\ \hline
%Zeile
	\textbf{Vorbedingungen}       &
	\begin{tabular}{l}
		-	Der Akteur muss im System angemeldet sein (Use Case 1 erfüllt). \\


	\end{tabular}		
\\ \hline
%Zeile
	\textbf{Input Information}             &        
	\begin{tabular}{l}
		-	Liste Nachbestellung Zentrallager
	\end{tabular}
\\ \hline
%Zeile
	\textbf{Ergebnisse}  &
	\begin{tabular}{l}
		Nachbestellungen werden aufgelistet.
	\end{tabular}
\\ \hline
%Zeile
	\textbf{Nachbedingung}				 &
	\begin{tabular}{l}
		-
	\end{tabular}
\\ \hline
%Zeile
	\textbf{Ablauf}            &
	\begin{tabular}{l}
		1.	Schaltfläche \grqq Nachbestellungen\grqq{} betätigen. \\
		2.	Produkt in Liste auswählen. \\
		3.	Schaltfläche \grqq Anzeigen\grqq{} betätigen.

	\end{tabular}
\\ \hline
%Zeile
	\textbf{Sonderfälle}			 &
	\begin{tabular}{l}
		- Es wird eine leere Liste angezeigt wenn keine Nachbestellungen ausgelöst worden sind.
	\end{tabular}
\\ \hline	
%Untere Abgrenzung
\end{tabular}
\label{tab:8usecase}
\caption{Use Case 8: Nachbestellung einsehen}
\end{table}