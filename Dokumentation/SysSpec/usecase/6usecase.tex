% !TEX root = Dokumentation_SysSpec.tex
\subsubsection{Use Case 6: Logbuch überwachen}
\begin{table}[H]
\begin{tabular}{ | p{0.22\textwidth} | p{0.68\textwidth} |} \hline
\rowcolor{gray!50}
%Titelzeile
	\textbf{Name}          &
	\begin{tabular}{l}
		\textbf{Logbuch überwachen}
	\end{tabular}
	\\ \hline
%Zeile
	\textbf{Kurzbeschreibung}			 &
	\begin{tabular}{l}
		Filialverwalter kann im Logbuch alle Aktivitäten betrachten.
	\end{tabular}
\\ \hline
%Zeile
	\textbf{Akteure}   		 &
	\begin{tabular}{l}
		Filialverwalter
	\end{tabular}
\\ \hline
%Zeile
	\textbf{Auslöser}              & 
	\begin{tabular}{l}
		Filialverwalter will Aktivitäten überprüfen.
	\end{tabular}
\\ \hline
%Zeile
	\textbf{Vorbedingungen}       &
	\begin{tabular}{l}
		-	Der Akteur muss im System angemeldet sein.\\
		-	Im Filial-Bestellsystem sind Aktivitäten abgelaufen .
		

	\end{tabular}		
\\ \hline
%Zeile
	\textbf{Input Information}             &        
	\begin{tabular}{l}
		-	Dokument Log öffnen.

	\end{tabular}
\\ \hline
%Zeile
	\textbf{Ergebnisse}  &
	\begin{tabular}{l}
		Filialverwalter bekommt File mit Aktivitäten.
	\end{tabular}
\\ \hline
%Zeile
	\textbf{Nachbedingung}				 &
	\begin{tabular}{l}
		File kann gespeichert werden.
	\end{tabular}
\\ \hline
%Zeile
	\textbf{Ablauf}            &
	\begin{tabular}{l}
		1.	File öffnen \\
		2.	Log betrachten oder speichern
		

	\end{tabular}
\\ \hline
%Zeile
	\textbf{Sonderfälle}			 &
	\begin{tabular}{l}
		- keine Aktivitäten im System -> File ist leer.\\
		
	\end{tabular}
\\ \hline	
%Untere Abgrenzung
\end{tabular}
\label{tab:6usecase}
\caption{Use Case 6: Logbuch überwachen}
\end{table}