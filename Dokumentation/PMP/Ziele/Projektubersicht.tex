% !TEX root = Dokumentation.tex
\section{Projektüberblick}
Das folgende Kapitel definiert die organisatorischen Aspekte im Zusammenhang mit dem Entwicklungsprojekt in APPE. Es umfasst dabei die Projektziele \&-resultate.
\subsection{Projektauftrag, Ziele \& Resultate}
Der Kunde hat ein Informatik-System für ein elektronisches Artikel-Bestellsystem für seine dezentralen Filialen bestellt.\\
Jede Filiale hat ein lokales Lager, welches in einer lokalen Datenbank (pro Filiale) verwaltet wird. Daneben existiert auch ein zentrales Lager, dieses wird in einem bestehenden System (Legacy System) verwaltet, welches über eine zur Verfügung gestellte Komponente einzubinden ist. Im zentralen Lager sind die Artikel entweder sofort lieferbar oder müssen nachbestellt werden.
\subsubsection{Ziele}
Mit dem Projekt sind dabei folgende Ziele zu erreichen:\\
\textbf{Funktionale Ziele / Anforderungen}
\begin{itemize}
\item Das Verkaufspersonal kann per Telefon, Schalterkontakt, Fax oder Mail eingegangene Bestellungen im System erfassen, d.h. es können Artikel aus einem Katalog auswählt, die Verfügbarkeit der Artikel geprüft und die Bestellung ausgeführt werden.
\item Auch wenn mehrere Benutzer einer Filiale gleichzeitig im System arbeiten, dürfen keine inkonsistenten Zustände entstehen (Bsp.: gleiches Exemplar mehrfach verkaufen)
\item Benutzer/innen des Systems müssen sich mit Name und Passwort identifizieren, ihnen werden danach entsprechende Rechte zugeteilt. (Benutzergruppen: SysAdmin, Filialleiter/in, Verkäufer/in, Datentypist/in)
\item Der Filialverwalter kann die Vorgänge im System (Fehler, wesentliche Geschäftsvorfälle) anhand von Logbucheinträgen laufend überwachen.
\item Der Filialleiter kann jederzeit die aktuellen Bestellungen, Nachbestellungen und Lieferungen aus der Zentrale einsehen
\item Der/die Datentypist/in kann im System angelieferte Artikel im Filiallager eintragen.
\item Das Verkaufspersonal und der Filialleiter können jederzeit den Zustand der Bestellungen einsehen. Eine Bestellung kann auch geändert oder annulliert werden.
\item Jegliche Exception und pro wesentlicher Business-Vorgang werden Logger-Einträge mitteils einem Logging-Framework aufgezeichnet.
\end{itemize}

\textbf{Organisatorische Ziele / Anforderungen}
\begin{itemize}
\item Alle Gruppenmitglieder müssen in der Softwareerstellung involviert sein. Die entsprechenden Verantwortlichkeiten sind in der Dokumentation ausgewiesen.
\end{itemize}

\textbf{Abgrenzungen \& Vorgaben}
\begin{itemize}
\item Die Buchhaltung inkl. Mahnwesen für die Rechnungsstellung/-behandlung erfolgt in einer separaten RW-Applikation mit Zugriff auf die lokale DB und ist nicht Teil des Auftrages.
\item Der Datenbankserver wird zur Verfügung gestellt.
\item Die Komponente für die Einbindung des Legacy Systems (Zentrallager) wird durch den Auftraggeber geliefert / zur Verfügung gestellt.
\end{itemize}

\subsubsection{Resultate}
Im Bereich der Softwareentwicklung sind entsprechende Resultate auszuweisen. Dabei werden in der Systemspezifikationen folgende Resultate beschrieben:
\begin{itemize}
\item Ein Kontext-Diagramm
\item Übersichtsdiagramm aller Use Cases inkl. einer Kurzbeschreibung der Use Cases. Detaillierte Beschreibung des Use Cases 'Bestellung ausführen'.
\item Konzeptionelles Datenmodell
\item Architektur-Modell(e) und nachvollziehbare Architekturentscheide
\item Schnittstellen-Spezifikation(en)
\end{itemize}

Im PMP werden dabei folgende Resultate abgebildet:
\begin{itemize}
\item Projektverantwortlichkeiten
\item Rahmenplan mit wichtigen Meilensteinen und Sprints
\item Dokumentation der Epics und Stories inkl. der Akzeptanzkriterien.
\item Testplan \& Testprotokoll
\end{itemize}

Auf applikatorischer Ebene werden folgende Softwaredeliverables erwartet:\\
Ein lauffähiges Programm, welches:
\begin{itemize}
\item in der vorgegebenen Projektstruktur auf GitLab verwaltet wird.
\item über einen automatisierten Build verfügt und auf dem Buildserver laufend integriert wird (CI)
\item automatisierte (Unit-)Tests ohne Fehler durchläuft und über eine begründete minimale Codeabdeckung verfügt
\end{itemize}

\subsection{Organisation, Rollen \& Zuständigkeiten}
Das Projektteam besteht aus vier Personen. Aufgrund dieser geringen Anzahl wird grundsätzlich keine hierarchische Organisationsform angestrebt. Primär werden Lösungen und Entscheidung mit einem gemeinsamen Konsens gefunden / gefällt. Bei unentschiedener Ausgangslage entscheidet der Projektleiter endgültig.
Die Rollen und Zuständigkeiten definieren die u.a. die Aufgaben und Verantwortungen einer Rolle und sind für unser Team wie folgt definiert:
% !TEX root = Dokumentation_PMP.tex
\begin{table}[H]
\begin{tabular}{ | p{0.22\textwidth} | p{0.19\textwidth} |  p{0.47\textwidth} |}
\hline \rowcolor{gray!50}
%Titelzeile
	\textbf{Rolle}          &
	\begin{tabular}{l}
		\textbf{Verantwortung}
	\end{tabular}	 	 &
	\begin{tabular}{l}
		\textbf{Aufgaben}
	\end{tabular}
	\\ \hline
%Zeile
	\textbf{Projektmanagement}			 &
	\begin{tabular}{l}
		Marco M.
	\end{tabular}	 	 			&
	\begin{tabular}{l}
		Ist für das Projektmanagement verantwortlich \\
		und führt Protokolle, Risikoliste und stellt \\
		bei Bedarf die externe Kommunikation sicher
	\end{tabular}
\\ \hline
%Zeile
	\textbf{Anforderungs-management}   		 &
	\begin{tabular}{l}
		Severin G.
	\end{tabular}	 	 			&
	\begin{tabular}{l}
		Klärt die funktionalen und nicht \\
		funktionalen Anforderungen ab
	\end{tabular}
\\ \hline
%Zeile
	\textbf{Dokument- \& Codeverwaltung}   		 &
	\begin{tabular}{l}
		Tobias K.
	\end{tabular}	 	 			&
	\begin{tabular}{l}
		Erstellung \& Verwaltung der Dokumentationen \\
		inkl. zentraler Ablage \& Versionierung \& \\
		Austausch Codesourcen mittels Git
	\end{tabular}
\\ \hline
%Zeile
	\textbf{Softwarearchitektur}              & 
	\begin{tabular}{l}
		Tobias K.
	\end{tabular}                 &
	\begin{tabular}{l}
		Plant und entwirft die SW-Architektur und \\
		definiert Schnittstellen. Prüft, ob die Archi-\\
		tektur (Layering, etc.) korrekt umgesetzt wurde
	\end{tabular}
\\ \hline
%Zeile
	\textbf{SW-Entwicklung Layer Data}       &
	\begin{tabular}{l}
		Ramon W.
	\end{tabular}	 	 &
	\begin{tabular}{l}
		Ist für die SW-Entwicklung des Layers Data \\
		verantwortlich und stellt dadurch die Daten-\\
		grundlage des Systems zur Verfügung. Erstellt \\
		und Unterhält die DB und die Anbindung mit \\
		dem O/R Mapper
	\end{tabular}
\\ \hline
%Zeile
	\textbf{SW-Entwicklung Layer Business}             &        
	\begin{tabular}{l}
		Marco M.
	\end{tabular}	 	 &
	\begin{tabular}{l}
		Ist für die SW-Entwicklung des Layers Business \\
		verantwortlich. Die Business-Schicht \\
		implementiert dabei die Geschäftsprozesse bzw. \\
		Use-Cases. Hat ebenfalls die Aufgabe als \\
		Vermittlungsschicht zwischen Daten- \& \\
		Präsentationsschicht zu agieren.
	\end{tabular}
\\ \hline
%Zeile
	\textbf{SW-Entwicklung Layer Presentation}  &
	\begin{tabular}{l}
		Severin G.
	\end{tabular}	 	 &
	\begin{tabular}{l}
		Ist für die SW-Entwicklung des Layers\\
		Presentation / GUI verantwortlich. Stellt die \\
		Funktionsblöcke \& Resultate grafisch dar. 
	\end{tabular}
\\ \hline
%Zeile
	\textbf{Test- \& QS-Management}				 &
	\begin{tabular}{l}
		Ramon W.
	\end{tabular}	 	 &
	\begin{tabular}{l}
		Definiert den Testplan und kontrolliert das\\
		Testing inkl. Protokolle. Ist ebenfalls für \\
		die Qualitätssicherung zuständig 
	\end{tabular}
\\ \hline	%Untere Abgrenzung
\end{tabular}
\end{table}