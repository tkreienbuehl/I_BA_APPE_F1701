% !TEX root = Dokumentation_PMP.tex
\section{Projektunterstützung}
Im Zentrum stehen hier Werkzeuge, welche das Team in ihrer Arbeit unterstützen sollen.
Ausserdem werden hier auch das Konfigurations-Management beschrieben, welche pro Sprint vorliegen.
\subsection{Tools für Entwicklung, Test \& Abnahme}
Im Rahmen des Moduls werden folgende Tools verwendet.
\begin{table}[H]
\begin{tabular}{ | p{0.30\textwidth} | p{0.70\textwidth} | }
\hline \rowcolor{gray!50}
%Titelzeile
	\textbf{Zweck} 		 &
	\textbf{Tool}
	\\	\hline
%Zeile
	Scrum-Tool 	&
	ScrumDo (Online)
	\\  \hline	%Untere Abgrenzung
%Zeile
	ERM-/ERD-Designer 		&
	MySQL Workbench v.6.3
	\\  \hline	%Untere Abgrenzung
%Zeile
	UML-Designer 		&
	Dia v0.97.2 \& MS Visio 2016
	\\  \hline	%Untere Abgrenzung
%Zeile
	Entwicklungsumgebung 		&
	Eclipse Neon 3
	\\  \hline	%Untere Abgrenzung
%Zeile
	Sourcecode Verwaltung 		&
	GitLab CE 8.14.2
	\\  \hline	%Untere Abgrenzung
%Zeile
	Sourcecode Style 		&
	CheckStyle / PMD / JaCoCo (Coverage)
	\\  \hline	%Untere Abgrenzung
%Zeile
	Buildserver 		&
	Maven, Jenkins v.1.558
	\\  \hline	%Untere Abgrenzung
%Zeile
	Test 		&
	JUnit / manuelle Integrations-Tests
	\\  \hline	%Untere Abgrenzung
%Zeile
	Dokumentation 		&
	\LaTeX  on GitHub
	\\  \hline	%Untere Abgrenzung
\end{tabular}
\label{tab:tooling}
\caption{Tools}
\end{table}


\subsection{Konfigurationsmanagement}
Jede Komponente wird versioniert. Sie kann nur mit definierten Versionen anderer Komponenten im Verbund korrekt zusammenarbeiten. Daher werden hier die Komponentenversionen pro Sprint (shippable Software) aufgelistet. Die Tests basieren auf diesen Versionen.
\subsubsection{Softwarekomponenten}
\begin{table}[H]
\begin{tabular}{ | p{0.20\textwidth} | p{0.10\textwidth} | p{0.10\textwidth} | p{0.10\textwidth} | p{0.10\textwidth} | }
\hline \rowcolor{gray!50}
%Titelzeile
	\textbf{Artefakt} 		 &
	\textbf{Sprint 1}	 	 &
	\textbf{Sprint 2}	 	 &
	\textbf{Sprint 3}		 &
	\textbf{Sprint 4}
	\\	\hline
%Zeile
	fbs\_business 	&
	v0.1 			&
	v0.2 			&
	v0.3 			&
	v1.0
	\\  \hline	%Untere Abgrenzung
%Zeile
	fbs\_client 		&
	v0.1 			&
	v0.2 			&
	v0.3 			&
	v1.0
	\\  \hline	%Untere Abgrenzung
%Zeile
	fbs\_data 		&
	v0.1 			&
	v0.2 			&
	v0.3 			&
	v1.0
	\\  \hline	%Untere Abgrenzung
%Zeile
	fbs\_model 		&
	v0.1 			&
	v0.2 			&
	v0.3 			&
	v1.0
	\\  \hline	%Untere Abgrenzung
\end{tabular}
\label{tab:config_sw}
\caption{Softwarekomponenten Konfigurationsmanagement}
\end{table}

\subsubsection{Dokumentation}
\begin{table}[H]
\begin{tabular}{ | p{0.30\textwidth} | p{0.10\textwidth} | p{0.10\textwidth} | p{0.10\textwidth} | p{0.10\textwidth} | }
\hline \rowcolor{gray!50}
%Titelzeile
	\textbf{Dokument} 		&
	\textbf{Sprint 1}	 	&
	\textbf{Sprint 2}	 	&
	\textbf{Sprint 3}		&
	\textbf{Sprint 4}
	\\	\hline
%Zeile
	Projektmanagementplan 	&
	v0.1 					&
	v0.2 					&
	v0.3 					&
	v1.0
	\\  \hline	%Untere Abgrenzung
%Zeile
	Systemspezifikation 	&
	v0.1 					&
	v0.2 					&
	v0.3 					&
	v1.0
	\\  \hline	%Untere Abgrenzung
\end{tabular}
\label{tab:config_dok}
\caption{Dokumentaiton Konfigurationsmanagement}
\end{table}
