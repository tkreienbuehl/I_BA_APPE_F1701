% !TEX root = Dokumentation_PMP.tex
Das Projektteam besteht aus vier Personen. Aufgrund dieser geringen Anzahl wird grundsätzlich keine hierarchische Organisationsform angestrebt. Primär werden Lösungen und Entscheidung mit einem gemeinsamen Konsens gefunden / gefällt. Bei unentschiedener Ausgangslage entscheidet der Projektleiter endgültig.
Die Rollen und Zuständigkeiten definieren die u.a. die Aufgaben und Verantwortungen einer Rolle und sind für unser Team wie folgt definiert:\\
\begin{table}[H]
\begin{tabular}{ | p{0.22\textwidth} | p{0.19\textwidth} |  p{0.47\textwidth} |}
\hline \rowcolor{gray!50}
%Titelzeile
	\textbf{Rolle}          &
	\begin{tabular}{l}
		\textbf{Verantwortung}
	\end{tabular}	 	 &
	\begin{tabular}{l}
		\textbf{Aufgaben}
	\end{tabular}
	\\ \hline
%Zeile
	\textbf{Projektmanagement}			 &
	\begin{tabular}{l}
		Marco M. \\
		Stv: Tobias K.
	\end{tabular}	 	 			&
	\begin{tabular}{l}
		Ist für das Projektmanagement verantwortlich \\
		und führt Protokolle, Risikoliste und stellt \\
		bei Bedarf die externe Kommunikation sicher
	\end{tabular}
\\ \hline
%Zeile
	\textbf{Anforderungs-management}   		 &
	\begin{tabular}{l}
		Severin G. \\
		Stv: Marco M.
	\end{tabular}	 	 			&
	\begin{tabular}{l}
		Klärt die funktionalen und nicht \\
		funktionalen Anforderungen ab
	\end{tabular}
\\ \hline
%Zeile
	\textbf{Dokument- \& Codeverwaltung}   		 &
	\begin{tabular}{l}
		Tobias K. \\
		Stv: Marco M.
	\end{tabular}	 	 			&
	\begin{tabular}{l}
		Erstellung \& Verwaltung der Dokumentationen \\
		inkl. zentraler Ablage \& Versionierung \& \\
		Austausch Codesourcen mittels Git
	\end{tabular}
\\ \hline
%Zeile
	\textbf{Softwarearchitektur}              & 
	\begin{tabular}{l}
		Tobias K. \\
		Stv: Ramon W.
	\end{tabular}                 &
	\begin{tabular}{l}
		Plant und entwirft die SW-Architektur und \\
		definiert Schnittstellen. Prüft, ob die Archi-\\
		tektur (Layering, etc.) korrekt umgesetzt wurde
	\end{tabular}
\\ \hline
%Zeile
	\textbf{SW-Entwicklung Layer Data}       &
	\begin{tabular}{l}
		Tobias K.\\
		Stv: Ramon W.
	\end{tabular}	 	 &
	\begin{tabular}{l}
		Ist für die SW-Entwicklung des Layers Data \\
		verantwortlich und stellt dadurch die Daten-\\
		grundlage des Systems zur Verfügung. Erstellt \\
		und Unterhält die DB und die Anbindung mit \\
		dem O/R Mapper
	\end{tabular}
\\ \hline
%Zeile
	\textbf{SW-Entwicklung Layer Business}             &        
	\begin{tabular}{l}
		Marco M. \\
		Stv: Severin G.
	\end{tabular}	 	 &
	\begin{tabular}{l}
		Ist für die SW-Entwicklung des Layers Business \\
		verantwortlich. Die Business-Schicht \\
		implementiert dabei die Geschäftsprozesse bzw. \\
		Use-Cases. Hat ebenfalls die Aufgabe als \\
		Vermittlungsschicht zwischen Daten- \& \\
		Präsentationsschicht zu agieren.
	\end{tabular}
\\ \hline
%Zeile
	\textbf{SW-Entwicklung Layer Presentation}  &
	\begin{tabular}{l}
		Ramon W. \\
		Stv: Marco M.
	\end{tabular}	 	 &
	\begin{tabular}{l}
		Ist für die SW-Entwicklung des Layers\\
		Presentation / GUI verantwortlich. Stellt die \\
		Funktionsblöcke \& Resultate grafisch dar. 
	\end{tabular}
\\ \hline
%Zeile
	\textbf{Test- \& QS-Management}				 &
	\begin{tabular}{l}
		Severin G. \\
		Stv: Marco M.
	\end{tabular}	 	 &
	\begin{tabular}{l}
		Definiert den Testplan und kontrolliert das\\
		Testing inkl. Protokolle. Ist ebenfalls für \\
		die Qualitätssicherung zuständig 
	\end{tabular}
\\ \hline	%Untere Abgrenzung
\end{tabular}
\end{table}
Im Bereich SW-Entwicklung entsprechend die Verantwortlichen inkl. Stellvertretung auch als primäre Entwickler der Layer.