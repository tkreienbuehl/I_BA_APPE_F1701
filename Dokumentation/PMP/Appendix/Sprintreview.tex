% !TEX root = Dokumentation_PMP.tex
\subsection{Berichte Sprintreview \& Meilsteinberichte}
Die vorliegenden Sprintreview sind nur stichwortartig festgehalten.
\subsubsection{Sprintreview \& MS1 - 07.04.2017}
\begin{tabular}{|l|l|} \rowcolor{gray!50}
	\hline 
	Datum & 07.04.2017 14:35-14:50 \\ 
	\hline 
	Teilnehmer &
	\begin{tabular}{l}
		J. Hofstetter (PO) \\
		S. Gmür \\
		R. Wyss\\
		M. Moro \\
	\end{tabular} \\
	\hline 
	Absenzen & T. Kreienbühl (krank) \\ 
	\hline 
\end{tabular} \\ \\
Die Iteration ist auf ScrumDo unter https://app.scrumdo.com/projects/appe-17fs-g13/\#/iteration/199049/board erreichbar.
Im ersten Sprint wurden ebenfalls noch administrative Tasks wie Dokumentationsbasis etc. erfasst, um ScrumDo auch für die initialen Aufgaben zu gebrauchen. \\
Im Sprint 1 wurden folgende geplanten Stories umgesetzt:\\
\begin{tabular}{|l|l|l|} \rowcolor{gray!50}
	\hline
	\textbf{Story-ID} & \textbf{Storytitel} & \textbf{Bemerkung} \\
	\hline 
	I8-18 & 
	Liste aller Bestellungen anzeigen &
	\begin{tabular}{l}
		Aufgrund der Tatsache, dass der Datalayer noch nicht \\verfügbar ist, wurde ein Stub erstellt, welcher die 'Test-\\bestelldaten' zur Verfügung stellt. Diese Daten konnten\\ erfolgreich im JavaFX GUI angezeigt werden.
	\end{tabular} \\ 
	\hline
	I8-10 & 
	Architekturdesign &
	\begin{tabular}{l}
		Die Layer- \& Tierarchitektur wurde im Team festgelegt. \\Klassendiagramm für 'Order' ist erstellt. \\Entsprechend notwendige Schnittstellen und DTOs sind \\definiert und in der SysSpec festgehalten. Die weiteren \\Klassendiagramme wie 'Article', etc. werden \\kontinuierlich erstellt / ergänzt.
	\end{tabular} \\
	\hline
	I8-15 & 
	DB-Schema erstellen &
	\begin{tabular}{l}
		Schema ist erstellt und die DDL-Skripts liegen vor. \\Die Dummydaten sind ebenfalls vorhanden und aktuell \\für Testzwecke in der Datenbank. Eine Abnahme durch \\R. Christen steht noch aus.
	\end{tabular} \\
	\hline
	I8-11 & 
	UseCase-Definitionen &
	\begin{tabular}{l}
		Die geforderten UseCases wurden entsprechend \\beschrieben (und UC 'Bestellung erfassen' detailliert) \\und mittels Übersichtsdiagramm aufgezeigt. \\Das PMP-Dokument wurde dahingehend ergänzt.
	\end{tabular} \\
	\hline
	I8-13 & 
	GUI-Design &
	\begin{tabular}{l}
		Die FXML-Pages (JavaFX) wurden für die Fälle 'Login', \\'Orderoverview', 'Orderdetail','NewOrder', 'EditOrder', \\'Supply', 'Logs' erstellt
	\end{tabular} \\
	\hline
	I8-16 & 
	Allgemeine administrative Arbeiten &
	\begin{tabular}{l}
		Dokumentationsbasis für SysSpec \& PMP inkl. \\Inhaltsgerüst sind erstellt. Im PMP sind die Risikoliste \\inkl. Massnahmen und die Anforderungen vorhanden.
	\end{tabular} \\
	\hline
\end{tabular}
\\ \\Die folgenden Stories konnte nicht abgeschlossen werden: \\
\begin{tabular}{|l|l|l|} \rowcolor{gray!50}
	\hline
	\textbf{Story-ID} & \textbf{Storytitel} & \textbf{Bemerkung} \\
	\hline 
	I8-2 & 
	Bestellung einsehen &
	\begin{tabular}{l}
		Die Anzeige auf grafischer Ebene funktioniert und die \\notwendigen Implementationen auf Layer Client und \\Business sind vorhanden.\\ Aufgrund der Verbindungsprobleme an die Datenbank \\wird diese Story in Sprint 2 weitergeführt.
	\end{tabular} \\
	\hline
	I8-17 & 
	O/R-Mapper &
	\begin{tabular}{l}
		Die Auswahl des O/R-Mappers konnte ausgewählt werden. \\Jedoch konnte die Implementierung noch nicht erfolgreich \\umgesetzt werden. Es bestehen noch Verbindungsprobleme,\\ welche im folgenden Sprint genauer betrachtet werden.
	\end{tabular} \\
	\hline
\end{tabular}
\clearpage

\subsubsection{Sprintreview \& MS2 - 28.04.2017}
\begin{tabular}{|l|l|} \rowcolor{gray!50}
	\hline 
	Datum & 28.04.2017 14:35-14:50 \\ 
	\hline 
	Teilnehmer &
	\begin{tabular}{l}
		J. Hofstetter (PO) \\
		T. Kreienbühl \\
		R. Wyss\\
		M. Moro \\
	\end{tabular} \\
	\hline 
	Absenzen & S. Gmür (Unfall) \\ 
	\hline 
\end{tabular} \\ \\
Die Iteration 2 ist auf ScrumDo unter https://app.scrumdo.com/projects/appe-17fs-g13/\#/iteration/200296/board erreichbar.
Im zweiten Sprint lag der Fokus im Datalayer, damit ein Durchstich mit dem UseCase 'Bestellung einsehen' und neue Bestellungen erfasst und auch bearbeitet werden können.
Die Anbindung dese O/R Mappers stellte eine Herausforderung dar, da die Konfiguration von JPA mittels EclipseLink in Eclipse nicht korrekt funktionierte. Der Durchstich konnte dabei mit Hilfe von R. Christen erreicht werden. \\
Aufgrund der Tatsache, dass der Durchstich bis zum Sprintreview noch nicht vorlag, wurde beschlossen umgehend mit R. Christen Kontakt aufzusuchen und die betroffenen Stories neu zu kalkulieren und in Sprint 3 zu übernehmen.\\

Im Sprint 2 wurden folgende geplanten Stories umgesetzt:\\
\begin{tabular}{|l|l|l|} \rowcolor{gray!50}
	\hline
	\textbf{Story-ID} & \textbf{Storytitel} & \textbf{Bemerkung} \\
	\hline 
	I8-2 & 
	Bestellung einsehen &
	\begin{tabular}{l}
		Die Story wurde aus Sprint 1 übernommen, da die Datenbankanbindung \\noch nicht funktionierte. Kurz nach dem Sprintreview konnte der\\ Durchstich erfolgreich getestet werden und die Bestellungen\\ eingesehen werden.
	\end{tabular} \\ 
	\hline
		I8-4 & 
	Bestellung annullieren &
	\begin{tabular}{l}
		Es gilt die gleiche Bemerkung wie bei I8-2. Es besteht jedoch noch die\\ Einschränkung, dass die zurückgebuchten Artikel nicht korrekt gebucht\\ werden. Dieser Aspekt wird aber separater Task behandelt.
	\end{tabular} \\ 
	\hline
	I8-1 & 
	Bestellung erfassen &
	\begin{tabular}{l}
		Es gilt die gleiche Bemerkung wie bei I8-2.
	\end{tabular} \\ 
	\hline
\end{tabular}
\\ \\Die folgenden Stories konnte nicht abgeschlossen werden: \\
\begin{tabular}{|l|l|l|} \rowcolor{gray!50}
	\hline
	\textbf{Story-ID} & \textbf{Storytitel} & \textbf{Bemerkung} \\
	\hline 
	I8-3 & 
	Bestellungen bearbeiten &
	\begin{tabular}{l}
		Aus zeitlichen Engpässen und Krankheiten wurde diese Story \\in Sprint 3 verschoben. 
	\end{tabular} \\
	\hline
\end{tabular}
\clearpage

\subsubsection{Sprintreview \& MS3 - 12.05.2017}
\begin{tabular}{|l|l|} \rowcolor{gray!50}
	\hline 
	Datum & 12.05.2017 14:35-14:50 \\ 
	\hline 
	Teilnehmer &
	\begin{tabular}{l}
		J. Hofstetter (PO) \\
		T. Kreienbühl \\
		R. Wyss\\
		M. Moro \\
	\end{tabular} \\
	\hline 
	Absenzen & S. Gmür (Unfall) \\ 
	\hline 
\end{tabular} \\ \\
Die Iteration 3 ist auf ScrumDo unter https://app.scrumdo.com/projects/appe-17fs-g13/\#/iteration/200298/board erreichbar.\\
Im dritten Sprint lag der Fokus im Bereich der Bearbeitung von Bestellungen (Anpassung Bestellstatus, Artikelmenge, etc.). Zusätzlich wird die Authentifizierung \& Autorisierung am System umgesetzt. Ebenfalls werden die Logbucheinträge (Anbindung Logger) realisiert. Bei zeitlichen Reserven wird die Story 'Nachbestellung auslösen' in Angriff genommen.\\

Im Sprint 3 wurden folgende geplanten Stories umgesetzt:\\
\begin{tabular}{|l|l|l|} \rowcolor{gray!50}
	\hline
	\textbf{Story-ID} & \textbf{Storytitel} & \textbf{Bemerkung} \\
	\hline 
	I8-3 & 
	Bestellung bearbeiten &
	\begin{tabular}{l}
		Die Story wurde aus zeitlichen und personellen Gründen aus Sprint 2\\ übernommen. Die Bestelländerungen können über das GUI vorge-\\nommen werden (Änderungen Artikelmenge, Status) und werden auf \\dem Business-Layer auf Gültigkeit (z.B Lagerbestand und Auslösung \\Rechnung) überprüft.
	\end{tabular} \\ 
	\hline
	I8-9 & 
	Am System anmelden &
	\begin{tabular}{l}
		Beim Start der Applikation werden nun die Zugangsdaten benötigt. \\Anhand des Users bzw. dessen Rolle wird dem Benutzer direkt die \\korrekte Page angezeigt. Auf jeder Page wird mittels Context-Check \\geprüft, ob der User diese Page verwenden darf. Die Benutzer-\\informationen sind in DB gespeichert.
	\end{tabular} \\ 
	\hline
	I8-6 & 
	Logbuch überwachen &
	\begin{tabular}{l}
		Es wurde für den Filialverwalter eine Page für die Logeinträge \\(Verweis auf Logfile) erstellt. Die Business-Vorgänge und Exception\\ werden in eine Textfile protokolliert.
	\end{tabular} \\ 
	\hline
\end{tabular}
\\ \\Die folgenden Stories konnte nicht abgeschlossen werden: \\
\begin{tabular}{|l|l|l|} \rowcolor{gray!50}
	\hline
	\textbf{Story-ID} & \textbf{Storytitel} & \textbf{Bemerkung} \\
	\hline 
	I8-7 & 
	Nachbestellung auslösen &
	\begin{tabular}{l}
		Im Sprint 3 konnten erst die Schnittstellen und die GUI-Page \\erstellt werden. Die Implementation steht noch aus. 
	\end{tabular} \\
	\hline
\end{tabular}
\clearpage

\subsubsection{Sprintreview \& MS4 - 01.06.2017}
\begin{tabular}{|l|l|} \rowcolor{gray!50}
	\hline 
	Datum & 01.06.2017 14:35-14:50 \\ 
	\hline 
	Teilnehmer &
	\begin{tabular}{l}
		J. Hofstetter (PO) \\
		T. Kreienbühl \\
		R. Wyss\\
		M. Moro \\
		S. Gmür \\
	\end{tabular} \\
	\hline 
	Absenzen & keine \\ 
	\hline 
\end{tabular} \\ \\
Die Iteration 4 ist auf ScrumDo unter https://app.scrumdo.com/projects/appe-17fs-g13/\#/iteration/200299/board erreichbar.\\
Die noch ausstehenden Pflicht-Stories konnten soweit abgeschlossen werden. Die Story für den Ausbau für Multi-Tiering wurde zurückgestellt, da diese Anpassungen nicht innerhalb der verfügbaren Zeit bewältigt werden können. Die Entwicklungsarbeiten sind hiermit für Release 1.0 beendet.\\

Im Sprint 4 wurden folgende geplanten Stories umgesetzt:\\
\begin{tabular}{|l|l|l|} \rowcolor{gray!50}
	\hline
	\textbf{Story-ID} & \textbf{Storytitel} & \textbf{Bemerkung} \\
	\hline 
	I8-5 & 
	Wareneingang erfassen &
	\begin{tabular}{l}
		Die Story konnte erfolgreich abgeschlossen werden. \\
		Angelieferte Artikel werden mengentechnisch im Filiallager \\
		angepasst / erhöht.
	\end{tabular} \\ 
	\hline
	I8-8 & 
	Lieferstatus von Nachbestellung einsehen &
	\begin{tabular}{l}
		Der Lieferstatus einer Nachbestellung wird in der 'Übersicht Nachbestellungen' angezeigt.
	\end{tabular} \\ 
	\hline
	I8-7 & 
	Nachbestellung auslösen &
	\begin{tabular}{l}
		Die Nachbestellung wurde beim Unterschreiten der Mindestmenge von 2 Elementen automatisch ausgelöst.\\
		Bei Menge 8 an Lager und einer Bestellung von Menge 10 wird eine Nachbestellung von Menge 22 (2 im Minus + 20 zusätzlich) ausgelöst und über Data-Layer zusätzlich persistiert.
	\end{tabular} \\ 
	\hline
\end{tabular}
\\ \\Die folgenden Stories konnte nicht abgeschlossen werden: \\
\begin{tabular}{|l|l|l|} \rowcolor{gray!50}
	\hline
	\textbf{Story-ID} & \textbf{Storytitel} & \textbf{Bemerkung} \\
	\hline 
	I8-19 & 
	Tier-Architektur einbinden &
	\begin{tabular}{l}
		Im Sprint 4 konnte aus zeitlichen Gründen der Multi-Tier-\\Architektur-Ansatz nicht mehr verfolgt werden. Die Applikation \\wird somit bis Release 1.0.0 als Single-Tier-Applikation betrieben. 
	\end{tabular} \\
	\hline
\end{tabular}