% !TEX root = Dokumentation_PMP.tex
\subsection{Berichte Sprintreview \& Meilsteinberichte}
Die vorliegenden Sprintreview sind nur stichwortartig festgehalten.
\subsubsection{Sprintreview \& MS1 - 07.04.2017}
\begin{tabular}{|l|l|} \rowcolor{gray!50}
	\hline 
	Datum & 07.04.2017 14:35-14:50 \\ 
	\hline 
	Teilnehmer &
	\begin{tabular}{l}
		J. Hofstetter (PO) \\
		S. Gmür \\
		R. Wyss\\
		M. Moro \\
	\end{tabular} \\
	\hline 
	Absenzen & T. Kreienbühl (krank) \\ 
	\hline 
\end{tabular} \\ \\
Die Iteration ist auf ScrumDo unter https://app.scrumdo.com/projects/appe-17fs-g13/\#/iteration/199049/board erreichbar.
Im ersten Sprint wurden ebenfalls noch administrative Tasks wie Dokumentationsbasis etc. erfasst, um ScrumDo auch für die initialen Aufgaben zu gebrauchen. \\
Im Sprint 1 wurden folgende geplanten Stories umgesetzt:\\
\begin{tabular}{|l|l|l|} \rowcolor{gray!50}
	\hline 
	I8-18 & 
	Liste aller Bestellungen anzeigen &
	\begin{tabular}{l}
		Aufgrund der Tatsache, dass der Datalayer noch nicht verfügbar \\ ist, wurde ein Stub erstellt, welcher die 'Testbestelldaten' zur \\Verfügung stellt. Diese Daten konnten erfolgreich im JavaFX GUI \\angezeigt werden.
	\end{tabular} \\ 
	\hline
\end{tabular}
Die folgenden Stories konnte nicht abgeschlossen werden:

\subsubsection{Sprintreview \& MS2 - 28.04.2017}
\begin{tabular}{|l|l|} \rowcolor{gray!50}
	\hline 
	Datum & 28.04.2017 14:35-14:50 \\ 
	\hline 
	Teilnehmer &
	\begin{tabular}{l}
		J. Hofstetter (PO) \\
		T. Kreienbühl \\
		R. Wyss\\
		M. Moro \\
	\end{tabular} \\
	\hline 
	Absenzen & S. Gmür (Unfall) \\ 
	\hline 
\end{tabular} 
\subsubsection{Sprintreview \& MS3 - 12.05.2017}
\subsubsection{Sprintreview \& MS4 - 01.06.2017}