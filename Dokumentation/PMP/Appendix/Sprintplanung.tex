% !TEX root = Dokumentation_PMP.tex
\subsection{Sprintplanungen}
In diesem Kapitel wird aufgezeigt, welche Stories inkl. Akzeptanzkriterien in welchem Sprint eingeplant wurden und welche UseCases diese Stories betreffen. 
\subsubsection{Sprint 1}
Die Ziele im ersten Sprint liegen im Bereich der Definition (DB, UseCases und GUI) sowie der Anzeige aller vorhandenen Bestellungen.\\
\begin{tabular}{|l|l|l|l|l} \rowcolor{gray!50}
	\hline
	\textbf{Story-ID} & \textbf{Storytitel} & \textbf{UseCases} & \textbf{Akzeptanzkriterien}\\
	\hline 
	I8-18 &
	\begin{tabular}{l}
		Liste aller Bestell-\\ungen anzeigen
	\end{tabular} & 
	\begin{tabular}{l}
		UC 2
	\end{tabular} &
	\begin{tabular}{l}
		- PL: Alle vordefinierten Bestellungen werden in\\ JavaFX GUI aufgelistet\\
		- DL: Aufgrund Abhängigkeit zu I8-17, I8-15 ggf. \\als Stub andernfalls über DB-Anbindung.
	\end{tabular} \\ 
	\hline 
	I8-10 & 
	\begin{tabular}{l}
	Architekturdesign
	\end{tabular} & 
	\begin{tabular}{l}
		Alle
	\end{tabular} &
	\begin{tabular}{l}
		- Layerarchitektur ist einer ersten Version vorhanden\\
		- Klassendiagramm für mind. UC 'Bestellungen\\ einsehen'\\
		- Definition der Interfaces \& DTOs zwischen BL-DL\\ und BL-PL
	\end{tabular} \\ 
	\hline 
	I8-12 & 
	\begin{tabular}{l}
	Bestellung einsehen
	\end{tabular} & 
	\begin{tabular}{l}
		UC 2
	\end{tabular} &
	\begin{tabular}{l}
		- PL: Anzeige der Details einer Bestellung in grafischer Oberfläche\\
		- DL: Bestellungen werden via OR-Mapper aus Datenbank gelesen
	\end{tabular} \\ 
	\hline 
	I8-17 & 
	\begin{tabular}{l}
	O/R-Mapper
	\end{tabular} & 
	\begin{tabular}{l}
		Alle, ausser \\
		UC 6
	\end{tabular} &
	\begin{tabular}{l}
		- DL: Auswahl und Ausarbeitung des O/R-Mapper\\ sowie Anbindung der DB
	\end{tabular} \\ 
	\hline 
	I8-15 & 
	\begin{tabular}{l}
	DB-Schema erstellen
	\end{tabular} & 
	\begin{tabular}{l}
		Alle
	\end{tabular} &
	\begin{tabular}{l}
		- Modell  in erster Version vorliegend (noch nicht\\ zwingend abgenommen - siehe Testat)\\
		- SQL-DB gem. Modell aufgesetzt\\
		- Dummy-Daten sind als SQL-Skript vorhanden und\\ können für Testzwecke importiert werden
	\end{tabular} \\ 
	\hline 
	I8-11 & 
	\begin{tabular}{l}
	UseCase-Definitionen
	\end{tabular} & 
	\begin{tabular}{l}
		keine
	\end{tabular} &
	\begin{tabular}{l}
		- zu allen UseCases gem. Übersicht und \\Anforderungsliste sind folgende Punkte beschrieben:\\
		-- Kurzbeschrieb und bei UC "Bestellung" detaillierter\\
		-- Akteure, Auslöser, Precondition, Input, Ergebnisse \\und ggf. Sonderfälle\\
		-- Ablauf
	\end{tabular} \\ 
	\hline 
	I8-13 & 
	\begin{tabular}{l}
	GUI-Design
	\end{tabular} & 
	\begin{tabular}{l}
		Alle
	\end{tabular} &
	\begin{tabular}{l}
		- PL: Alle Views / Pages für Login, Detailorder, \\OrderOverview, New- /EditOrder, Supply sind in \\JavaFX abgebildet.
	\end{tabular} \\ 
	\hline 
	I8-16 & 
	\begin{tabular}{l}
	Allgemeine administ-\\rative Arbeiten
	\end{tabular} & 
	\begin{tabular}{l}
		keine
	\end{tabular} &
	\begin{tabular}{l}
		- Dokument für SysSpec mit Inhaltsübersicht liegt vor\\
		- PMP mit Organisation, Risikoliste, Anforderungen \\ist in einer ersten Version vorhanden
	\end{tabular} \\ 
\end{tabular}\\
\clearpage
\subsubsection{Sprint 2}
Im zweiten Sprint lag das Ziel im Bereich des Durchstiches. Dadurch soll es möglich sein Bestellungen einzusehen, bearbeiten und zu erfassen\\
\begin{tabular}{|l|l|l|l|l} \rowcolor{gray!50}
	\hline
	\textbf{Story-ID} & \textbf{Storytitel} & \textbf{UseCases} & \textbf{Akzeptanzkriterien}\\
	\hline 
	I8-2 &
	\begin{tabular}{l}
		Bestellung einsehen
	\end{tabular} & 
	\begin{tabular}{l}
		UC 2
	\end{tabular} &
	\begin{tabular}{l}
		- PL: Anzeige der Details einer Bestellung in \\grafischer Oberfläche\\
		- DL: Bestellungen werden via OR-Mapper \\aus Datenbank gelesen
	\end{tabular} \\ 
	\hline 
	I8-4 & 
	\begin{tabular}{l}
		Bestellung annullieren
	\end{tabular} & 
	\begin{tabular}{l}
		UC 4
	\end{tabular} &
	\begin{tabular}{l}
		- PL: Die Annullierung kann via grafische \\Oberfläche ausgelöst werden.\\
		- BL: Die Datenänderung wird auf Gültigkeit \\und Vor/Nachbedingungen geprüft.\\
		- DL: Die Datenänderung wird via OR-Mapper \\in die DB geschrieben
	\end{tabular} \\ 
	\hline 
	I8-3 & 
	\begin{tabular}{l}
		Bestellung bearbeiten
	\end{tabular} & 
	\begin{tabular}{l}
		UC 4
	\end{tabular} &
	\begin{tabular}{l}
		- PL: Über eine grafische Oberfläche \\können Bestellungsänderungen eingegeben werden\\
		- BL: Die Änderungen werden auf Gültigkeit \\validiert\\
		-- Artikel bekannt\\
		-- Lagerbestand genug hoch\\
		- BL: Änderungen werden an Filiallager und \\Buchhaltung gemeldet\\
		- DL: Die Änderungen werden via DB-Mapper \\in die DB geschrieben
	\end{tabular} \\ 
	\hline 
	I8-1 & 
	\begin{tabular}{l}
		Bestellung erfassen
	\end{tabular} & 
	\begin{tabular}{l}
		UC 3
	\end{tabular} &
	\begin{tabular}{l}
		- PL: Eingabe der Daten via Formular in grafi-\\scher Oberfläche\\
		- BL: Validierung der Eingaben:\\
		-- Bekannter Artikel\\
		-- Artikel vorhanden (lokal, Zentrallager)\\
		-- Kunde hat Mahnungen offen\\
		-- ohne Prüfung Zentrallagerbestand.\\
		-- Auslösen von Bestellbestätigung\\
		- DL: Persistierung der Eingabe in die Datenbank \\via OR-Mapper, inkl. Lagerbestandsanpassung\\
	\end{tabular} \\ 
	\hline 
\end{tabular}\\
\clearpage
\subsubsection{Sprint 3}
Im Sprint 3 soll es möglich sein vorhandene Bestellungen zu bearbeiten sowie die Anmeldung \& Logbuch zu implementieren\\
\begin{tabular}{|l|l|l|l|l} \rowcolor{gray!50}
	\hline
	\textbf{Story-ID} & \textbf{Storytitel} & \textbf{UseCases} & \textbf{Akzeptanzkriterien}\\
	\hline 
	I8-3 & 
	\begin{tabular}{l}
		Bestellung bearbeiten
	\end{tabular} & 
	\begin{tabular}{l}
		UC 4
	\end{tabular} &
	\begin{tabular}{l}
		- PL: Über eine grafische Oberfläche \\können Bestellungsänderungen eingegeben werden\\
		- BL: Die Änderungen werden auf Gültigkeit \\validiert\\
		-- Artikel bekannt\\
		-- Lagerbestand genug hoch\\
		- BL: Änderungen werden an Filiallager und \\Buchhaltung gemeldet\\
		- DL: Die Änderungen werden via DB-Mapper \\in die DB geschrieben
	\end{tabular} \\ 
	\hline 
	I8-9 &
	\begin{tabular}{l}
		Am System anmelden
	\end{tabular} & 
	\begin{tabular}{l}
		UC 1
	\end{tabular} &
	\begin{tabular}{l}
		- PL: Über ein Anmeldefeld am System anmelden\\
		- BL: Verifizierung der Benutzerdaten \& Zuweisung \\und Überprüfung der Benutzerdaten (Berechtigungs-\\konzept (Authorisation, Authentication)\\
		- DL: Die Zugangsdaten werden aus der Datenbank \\verifiziert
	\end{tabular} \\ 
	\hline 
	I8-6 & 
	\begin{tabular}{l}
		Logbuch überwachen
	\end{tabular} & 
	\begin{tabular}{l}
		UC 6
	\end{tabular} &
	\begin{tabular}{l}
		- PL: Anzeige von Logdaten\\
		- BL: Kategorisierung von Logdaten\\
		-- Fehler (Exceptions)\\
		-- Relevante Vorgänge\\
		- DL: Persistierung der Logdaten
	\end{tabular} \\ 
	\hline 

	I8-7 & 
	\begin{tabular}{l}
		Nachbestellung auslösen
	\end{tabular} & 
	\begin{tabular}{l}
		UC 7
	\end{tabular} &
	\begin{tabular}{l}
		- BL: Bei Unterschreitung eines Lagerbestands wird \\eine Nachbestellung zum Zentrallager gesendet.\\
		- DL: Die Nachbestellung ist in der Datenbank \\gespeichert
	\end{tabular} \\ 
	\hline 
\end{tabular}\\
\clearpage
\subsubsection{Sprint 4}
Im letzten Sprint werden die Aspekte der Nachbestellung und Wareneingang implementiert und bei zeitlicher Reserve wird die Tier-Separierung zwischen Presentation- \& Businesslayer in Betracht gezogen.\\
\begin{tabular}{|l|l|l|l|l} \rowcolor{gray!50}
	\hline
	\textbf{Story-ID} & \textbf{Storytitel} & \textbf{UseCases} & \textbf{Akzeptanzkriterien}\\
	\hline 
	I8-7 & 
	\begin{tabular}{l}
		Nachbestellung auslösen
	\end{tabular} & 
	\begin{tabular}{l}
		UC 7
	\end{tabular} &
	\begin{tabular}{l}
		- BL: Bei Unterschreitung eines Lagerbestands wird \\eine Nachbestellung zum Zentrallager gesendet.\\
		- DL: Die Nachbestellung ist in der Datenbank \\gespeichert
	\end{tabular} \\ 
	\hline
	I8-5 & 
	\begin{tabular}{l}
		Wareneingang erfassen
	\end{tabular} & 
	\begin{tabular}{l}
		UC 5
	\end{tabular} &
	\begin{tabular}{l}
		- PL: Via Formular werden Wareneingänge im \\System erfasst.\\
		- BL: Die Eingaben werden auf Gültigkeit\\ validiert\\
		-- Artikel ist bekannt\\
		-- Artikelstückzahl ist gültig\\
		- DL: Die Daten werden via OR-Mapper in die \\Datenbank geschrieben
	\end{tabular} \\ 
	\hline 
	I8-8 &
	\begin{tabular}{l}
		Nachbestellung einsehen
	\end{tabular} & 
	\begin{tabular}{l}
		UC 8 \& 9
	\end{tabular} &
	\begin{tabular}{l}
		- PL: Über eine grafische Oberfläche kann die \\Nachbestellung inkl. Status aus der DB an-\\gesehen werden		
	\end{tabular} \\ 
	\hline 
	I8-19 & 
	\begin{tabular}{l}
		Tier-Architektur einbinden
	\end{tabular} & 
	\begin{tabular}{l}
		keine
	\end{tabular} &
	\begin{tabular}{l}
		- PL: kann auf einem separaten Tier betrieben \\werden\\
		- BL: Das Transaktionsmanagement ist sauber \\definiert und implementiert\\
	\end{tabular} \\ 
	\hline 
\end{tabular}\\\\
Nach Sprint 4 waren für Release 1.0 keine weiteren Stories im Backlog. Die Entwicklung für Release 1.0 ist hiermit abschliessend und Code-Freeze wurde auf den 26.05.2017 festgelegt. Korrekturen aufgrund Fehler in Tests sind noch erlaubt.
