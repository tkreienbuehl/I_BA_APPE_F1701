% !TEX root = Dokumentation_PMP.tex
\subsubsection{Integrationstest 6 - Bestellung einsehen}
\begin{table}[H]
\begin{tabular}{ | p{0.22\textwidth} | p{0.68\textwidth} |} \hline
\rowcolor{gray!50}
%Titelzeile
	\textbf{ID}						&
	\begin{tabular}{l}
		\textbf{I6}
	\end{tabular}
	\\ \hline
%Zeile
	\textbf{Bezeichnung}			&
	\begin{tabular}{l}
		Bestellung einsehen
	\end{tabular}
\\ \hline
%Zeile
	\textbf{Beschreibung}			&
	\begin{tabular}{l}
		Bestellungen können eingesehen
werden.
	\end{tabular}
\\ \hline
%Zeile
	\textbf{Akteure}				&
	\begin{tabular}{l}
		Filialleiter, Verkaufspersonal.
	\end{tabular}
\\ \hline
%Zeile
	\textbf{Vorbedingungen}			&
	\begin{tabular}{l}
		Offene Bestellung vorhanden.\\
		Erfolgreich an System angemeldet.\\
		Benutzergruppe zugewiesen.
	\end{tabular}		
\\ \hline
%Zeile
	\textbf{Ergebnis}				&
	\begin{tabular}{l}
		Bestellung wird angezeigt
	\end{tabular}
\\ \hline
%Zeile
	\textbf{Ergebnis bei Fehler}	&
	\begin{tabular}{l}
		- Fehlermeldung erscheint
	\end{tabular}
\\ \hline
%Zeile
	\textbf{Ablauf}					&
	\begin{tabular}{l}
		1. Bestellungen ansehen
	\end{tabular}
\\ \hline
	\textbf{Testdaten}				&
	\begin{tabular}{l}
		Test-Bestellung.
	\end{tabular}
\\ \hline	%Untere Abgrenzung
\end{tabular}
\end{table}