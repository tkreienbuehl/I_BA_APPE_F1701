% !TEX root = Dokumentation_PMP.tex
\section{Testplan}
\subsection{Testdesign \& Abläufe}
Dieses Kapitel hält Integrations- und Systemtests fest. \\
Unittests wurden womöglich aktiv eingesetzt, wenn die zeitlichen Aufwände nicht zu hoch sind. Die Entscheidung obliegt jedem einzelnen Entwickler. Es wurde im Team entschieden, dass GUI-Elemente /-Klassen bewusst nicht mittels Unit-Tests abgedeckt werden. \\
Das Team strebt daher keine hohe Codeabdeckung an und setzt als Schwellwert > 30\% Codeabdeckung summarisch über alle Projekte gem. Jenkins-Report. \\
Der Unterschied beider Testarten ist folgender: Integrationstest sind teilautomatisiert mittels JUnit. Diese Tests können in der Entwicklungsumgebung nach dem Setup der Testumgebung gestartet werden. Wohingegen Systemtests vollständig manuell sind.
\subsection{Integrationstests}
% !TEX root = Dokumentation.tex
\subsubsection{Integrationstest 1}
\begin{table}[H]
\begin{tabular}{ | p{0.22\textwidth} | p{0.68\textwidth} |} \hline
\rowcolor{gray!50}
%Titelzeile
	\textbf{ID}          &
	\begin{tabular}{l}
		\textbf{I1}
	\end{tabular}
	\\ \hline
%Zeile
	\textbf{Bezeichnung}			 &
	\begin{tabular}{l}
		Applikation \glqq{}Filialbestelsystem\grqq{} kann gestartet werden.
	\end{tabular}
\\ \hline
%Zeile
	\textbf{Beschreibung}   		 &
	\begin{tabular}{l}
		Die Applikation kann gestartet werden.
	\end{tabular}
\\ \hline
%Zeile
	\textbf{Akteure}              & 
	\begin{tabular}{l}
		Nutzer
	\end{tabular}
\\ \hline
%Zeile
	\textbf{Vorbedingungen}       &
	\begin{tabular}{l}
		keine
	\end{tabular}		
\\ \hline
%Zeile
	\textbf{Ergebnis}             &        
	\begin{tabular}{l}
		Integrationstest werden fehlerfrei abgeschlossen. \\
		Die Applikation ist erfolgreich gestartet. \\
		Die Anmelde-Maske wird gestartet. \\
	\end{tabular}
\\ \hline
%Zeile
	\textbf{Ergebnis bei Fehler}  &
	\begin{tabular}{l}
		- Exception\\
		- Unit Test failed
	\end{tabular}
\\ \hline
%Zeile
	\textbf{Ablauf}				 &
	\begin{tabular}{l}
		1. Applikation starten\\
		2. Ergebnis prüfen
	\end{tabular}
\\ \hline
	\textbf{Testdaten}            &
	\begin{tabular}{l}
		Keine benötigt.
	\end{tabular}
\\ \hline	%Untere Abgrenzung
\end{tabular}
\end{table}
% !TEX root = Dokumentation.tex
\subsubsection{Integrationstest 2}
\begin{table}[H]
\begin{tabular}{ | p{0.22\textwidth} | p{0.68\textwidth} |} \hline
\rowcolor{gray!50}
%Titelzeile
	\textbf{ID}          &
	\begin{tabular}{l}
		\textbf{I2}
	\end{tabular}
	\\ \hline
%Zeile
	\textbf{Bezeichnung}			 &
	\begin{tabular}{l}
		Anmeldung an Applikation funktioniert.
	\end{tabular}
\\ \hline
%Zeile
	\textbf{Beschreibung}   		 &
	\begin{tabular}{l}
		Die Anmeldung mit Username und Passwort funktioniert\\
		an der Applikation.
	\end{tabular}
\\ \hline
%Zeile
	\textbf{Akteure}              & 
	\begin{tabular}{l}
		Benutzer
	\end{tabular}
\\ \hline
%Zeile
	\textbf{Vorbedingungen}       &
	\begin{tabular}{l}
		Applikation erfolgreich gestartet
	\end{tabular}		
\\ \hline
%Zeile
	\textbf{Ergebnis}             &        
	\begin{tabular}{l}
		Anmeldung erfolgreich\\
Benutzeransicht nach Benutzergruppe erscheint
	\end{tabular}
\\ \hline
%Zeile
	\textbf{Ergebnis bei Fehler}  &
	\begin{tabular}{l}
		- Fehlermeldung erscheint\\
	\end{tabular}
\\ \hline
%Zeile
	\textbf{Ablauf}				 &
	\begin{tabular}{l}
		1. Username eingeben\\
		2. Passwort eingeben\\
		3. Anmeldung bestätigen
	\end{tabular}
\\ \hline
	\textbf{Testdaten}            &
	\begin{tabular}{l}
		Username, Passwort und Gruppenzugehörigkeit.
	\end{tabular}
\\ \hline	%Untere Abgrenzung
\end{tabular}
\end{table}
% !TEX root = Dokumentation_PMP.tex
\subsubsection{Integrationstest 3 - Erfassung Wareneingang}
\begin{table}[H]
\begin{tabular}{ | p{0.22\textwidth} | p{0.68\textwidth} |} \hline
\rowcolor{gray!50}
%Titelzeile
	\textbf{ID}						&
	\begin{tabular}{l}
		\textbf{I3}
	\end{tabular}
	\\ \hline
%Zeile
	\textbf{Bezeichnung}			&
	\begin{tabular}{l}
		Wareneingang kann erfasst werden.
	\end{tabular}
\\ \hline
%Zeile
	\textbf{Beschreibung}			&
	\begin{tabular}{l}
		Die Erfassung von Lieferungen ans Filiallager kann erfasst werden.
	\end{tabular}
\\ \hline
%Zeile
	\textbf{Akteure}				& 
	\begin{tabular}{l}
		Datentypist
	\end{tabular}
\\ \hline
%Zeile
	\textbf{Vorbedingungen}			&
	\begin{tabular}{l}
		Applikation erfolgreich gestartet\\
Anmeldung erfolgreich\\
Benutzergruppe Datentypist zugewiesen
	\end{tabular}		
\\ \hline
%Zeile
	\textbf{Ergebnis}				&        
	\begin{tabular}{l}
		Bestätigung der erfassten Lieferung.
	\end{tabular}
\\ \hline
%Zeile
	\textbf{Ergebnis bei Fehler}	&
	\begin{tabular}{l}
		- Fehlermeldung erscheint
	\end{tabular}
\\ \hline
%Zeile
	\textbf{Ablauf}					&
	\begin{tabular}{l}
		1. Formular ausfüllen\\
		2. Formular absenden\\
		3. Bestätigung erhalten
	\end{tabular}
\\ \hline
	\textbf{Testdaten}				&
	\begin{tabular}{l}
		Test-Lieferung.
	\end{tabular}
\\ \hline	%Untere Abgrenzung
\end{tabular}
\end{table}
% !TEX root = Dokumentation_PMP.tex
\subsubsection{Integrationstest 4 - Auto Nachbestellung}
\begin{table}[H]
\begin{tabular}{ | p{0.22\textwidth} | p{0.68\textwidth} |} \hline
\rowcolor{gray!50}
%Titelzeile
	\textbf{ID}						&
	\begin{tabular}{l}
		\textbf{I4}
	\end{tabular}
	\\ \hline
%Zeile
	\textbf{Bezeichnung}			&
	\begin{tabular}{l}
		Automatische Nachbestellung.
	\end{tabular}
\\ \hline
%Zeile
	\textbf{Beschreibung}			&
	\begin{tabular}{l}
		Bei Unterschreiten eines Mindestbestandes wird eine automatische\\
Bestellung ausgelöst.
	\end{tabular}
\\ \hline
%Zeile
	\textbf{Akteure}				& 
	\begin{tabular}{l}
		Filiallager
	\end{tabular}
\\ \hline
%Zeile
	\textbf{Vorbedingungen}			&
	\begin{tabular}{l}
		Lagerbestand von Artikel unterschreitet Mindestbestand.
	\end{tabular}		
\\ \hline
%Zeile
	\textbf{Ergebnis}				&        
	\begin{tabular}{l}
		Nachbestellung ausgelöst.
	\end{tabular}
\\ \hline
%Zeile
	\textbf{Ergebnis bei Fehler}	&
	\begin{tabular}{l}
		- Fehlermeldung erscheint\\
		- Meldung an Filialleiter
	\end{tabular}
\\ \hline
%Zeile
	\textbf{Ablauf}					&
	\begin{tabular}{l}
		1. Bestellung aufgegeben\\
		2. Lagerbestand unterschreitet Mindestbestand\\
		3. Bestellung ausgelöst
	\end{tabular}
\\ \hline
	\textbf{Testdaten}				&
	\begin{tabular}{l}
		Artikel mit Lagermenge an Mindestgrenze.
	\end{tabular}
\\ \hline	%Untere Abgrenzung
\end{tabular}
\end{table}
% !TEX root = Dokumentation.tex
\subsubsection{Integrationstest 5}
\begin{table}[H]
\begin{tabular}{ | p{0.22\textwidth} | p{0.68\textwidth} |} \hline
\rowcolor{gray!50}
%Titelzeile
	\textbf{ID}						&
	\begin{tabular}{l}
		\textbf{I5}
	\end{tabular}
	\\ \hline
%Zeile
	\textbf{Bezeichnung}			&
	\begin{tabular}{l}
		Nachbestellung einsehen.
	\end{tabular}
\\ \hline
%Zeile
	\textbf{Beschreibung}			&
	\begin{tabular}{l}
		Die offenen Nachbestellungen am Zentrallager können eingesehen\\
werden.
	\end{tabular}
\\ \hline
%Zeile
	\textbf{Akteure}				& 
	\begin{tabular}{l}
		Filialleiter
	\end{tabular}
\\ \hline
%Zeile
	\textbf{Vorbedingungen}			&
	\begin{tabular}{l}
		Nachbestellung offen.\\
		Erfolgreich an System angemeldet.\\
		Benutzergruppe zugewiesen.
	\end{tabular}		
\\ \hline
%Zeile
	\textbf{Ergebnis}				&
	\begin{tabular}{l}
		Nachbestellung wird angezeigt
	\end{tabular}
\\ \hline
%Zeile
	\textbf{Ergebnis bei Fehler}	&
	\begin{tabular}{l}
		- Fehlermeldung erscheint
	\end{tabular}
\\ \hline
%Zeile
	\textbf{Ablauf}					&
	\begin{tabular}{l}
		1. Nachbestellung ansehen
	\end{tabular}
\\ \hline
	\textbf{Testdaten}				&
	\begin{tabular}{l}
		Vorhandene Nachbestellung.
	\end{tabular}
\\ \hline	%Untere Abgrenzung
\end{tabular}
\end{table}
% !TEX root = Dokumentation.tex
\subsubsection{Integrationstest 6 - Bestellung einsehen}
\begin{table}[H]
\begin{tabular}{ | p{0.22\textwidth} | p{0.68\textwidth} |} \hline
\rowcolor{gray!50}
%Titelzeile
	\textbf{ID}						&
	\begin{tabular}{l}
		\textbf{I6}
	\end{tabular}
	\\ \hline
%Zeile
	\textbf{Bezeichnung}			&
	\begin{tabular}{l}
		Bestellung einsehen
	\end{tabular}
\\ \hline
%Zeile
	\textbf{Beschreibung}			&
	\begin{tabular}{l}
		Bestellungen können eingesehen
werden.
	\end{tabular}
\\ \hline
%Zeile
	\textbf{Akteure}				&
	\begin{tabular}{l}
		Filialleiter, Verkaufspersonal.
	\end{tabular}
\\ \hline
%Zeile
	\textbf{Vorbedingungen}			&
	\begin{tabular}{l}
		Offene Bestellung vorhanden.\\
		Erfolgreich an System angemeldet.\\
		Benutzergruppe zugewiesen.
	\end{tabular}		
\\ \hline
%Zeile
	\textbf{Ergebnis}				&
	\begin{tabular}{l}
		Bestellung wird angezeigt
	\end{tabular}
\\ \hline
%Zeile
	\textbf{Ergebnis bei Fehler}	&
	\begin{tabular}{l}
		- Fehlermeldung erscheint
	\end{tabular}
\\ \hline
%Zeile
	\textbf{Ablauf}					&
	\begin{tabular}{l}
		1. Bestellungen ansehen
	\end{tabular}
\\ \hline
	\textbf{Testdaten}				&
	\begin{tabular}{l}
		Test-Bestellung.
	\end{tabular}
\\ \hline	%Untere Abgrenzung
\end{tabular}
\end{table}
% !TEX root = Dokumentation_PMP.tex
\subsubsection{Integrationstest 6 - Bestellung annullieren}
\begin{table}[H]
\begin{tabular}{ | p{0.22\textwidth} | p{0.68\textwidth} |} \hline
\rowcolor{gray!50}
%Titelzeile
	\textbf{ID}						&
	\begin{tabular}{l}
		\textbf{I7}
	\end{tabular}
	\\ \hline
%Zeile
	\textbf{Bezeichnung}			&
	\begin{tabular}{l}
		Bestellung annullieren.
	\end{tabular}
\\ \hline
%Zeile
	\textbf{Zugehörige Userstory}			 &
	\begin{tabular}{l}
		Bestellung annulieren (I8-4)
	\end{tabular}
\\ \hline
%Zeile
	\textbf{Beschreibung}			&
	\begin{tabular}{l}
		Eine Bestellung kann annulliert werden.
	\end{tabular}
\\ \hline
%Zeile
	\textbf{Akteure}				&
	\begin{tabular}{l}
		Filialleiter, Verkaufspersonal.
	\end{tabular}
\\ \hline
%Zeile
	\textbf{Vorbedingungen}			&
	\begin{tabular}{l}
		Offene Bestellung vorhanden.\\
		Erfolgreich an System angemeldet.\\
		Benutzergruppe zugewiesen.
	\end{tabular}		
\\ \hline
%Zeile
	\textbf{Ergebnis}				&
	\begin{tabular}{l}
		Bestellung annulliert.\\
		Rechnungswesen notifiziert.
	\end{tabular}
\\ \hline
%Zeile
	\textbf{Ergebnis bei Fehler}	&
	\begin{tabular}{l}
		- Fehlermeldung erscheint
	\end{tabular}
\\ \hline
%Zeile
	\textbf{Ablauf}					&
	\begin{tabular}{l}
		1. Bestellung öffnen\\
		2. Bestellung annullieren.
		3. Rechnungswesen notifizieren.\\
		4. Filiallager notifizieren.\\
		5. Bestätigung.
	\end{tabular}
\\ \hline
	\textbf{Testdaten}				&
	\begin{tabular}{l}
		Test-Bestellung.
	\end{tabular}
\\ \hline	%Untere Abgrenzung
\end{tabular}
\end{table}
% !TEX root = Dokumentation.tex
\subsubsection{Integrationstest 8 - Bestellung editieren}
\begin{table}[H]
\begin{tabular}{ | p{0.22\textwidth} | p{0.68\textwidth} |} \hline
\rowcolor{gray!50}
%Titelzeile
	\textbf{ID}						&
	\begin{tabular}{l}
		\textbf{I7}
	\end{tabular}
	\\ \hline
%Zeile
	\textbf{Bezeichnung}			&
	\begin{tabular}{l}
		Bestellung editieren.
	\end{tabular}
\\ \hline
%Zeile
	\textbf{Beschreibung}			&
	\begin{tabular}{l}
		Eine Bestellung kann editiert werden.
	\end{tabular}
\\ \hline
%Zeile
	\textbf{Akteure}				&
	\begin{tabular}{l}
		Filialleiter, Verkaufspersonal.
	\end{tabular}
\\ \hline
%Zeile
	\textbf{Vorbedingungen}			&
	\begin{tabular}{l}
		Offene Bestellung vorhanden.\\
		Erfolgreich an System angemeldet.\\
		Benutzergruppe zugewiesen.
	\end{tabular}		
\\ \hline
%Zeile
	\textbf{Ergebnis}				&
	\begin{tabular}{l}
		Bestellung editiert.\\
		Rechnungswesen notifiziert.
	\end{tabular}
\\ \hline
%Zeile
	\textbf{Ergebnis bei Fehler}	&
	\begin{tabular}{l}
		- Fehlermeldung erscheint
	\end{tabular}
\\ \hline
%Zeile
	\textbf{Ablauf}					&
	\begin{tabular}{l}
		1. Bestellung öffnen.\\
		2. Bestellung editieren und bestätigen.
		3. Rechnungswesen notifizieren.\\
		4. Filiallager notifizieren.\\
		5. Bestätigung.
	\end{tabular}
\\ \hline
	\textbf{Testdaten}				&
	\begin{tabular}{l}
		Test-Bestellung.
	\end{tabular}
\\ \hline	%Untere Abgrenzung
\end{tabular}
\end{table}
% !TEX root = Dokumentation_PMP.tex
\subsubsection{Integrationstest 8 - Bestellung erfassen}
\begin{table}[H]
\begin{tabular}{ | p{0.22\textwidth} | p{0.68\textwidth} |} \hline
\rowcolor{gray!50}
%Titelzeile
	\textbf{ID}								&
	\begin{tabular}{l}
		\textbf{I7}
	\end{tabular}
	\\ \hline
%Zeile
	\textbf{Bezeichnung}					&
	\begin{tabular}{l}
		Bestellung erfassen.
	\end{tabular}
\\ \hline
%Zeile
	\textbf{Zugehörige Userstory}			 &
	\begin{tabular}{l}
		Bestellung erfassen (I8-1)
	\end{tabular}
\\ \hline
%Zeile
	\textbf{Beschreibung}					&
	\begin{tabular}{l}
		Eine Bestellung kann erfasst
werden.
	\end{tabular}
\\ \hline
%Zeile
	\textbf{Akteure}						&
	\begin{tabular}{l}
		Filialleiter, Verkaufspersonal.
	\end{tabular}
\\ \hline
%Zeile
	\textbf{Vorbedingungen}					&
	\begin{tabular}{l}
		Erfolgreich an System angemeldet.\\
		Benutzergruppe zugewiesen.\\
		Für Testkunde keine offenen Mahnungen vorhanden.
	\end{tabular}		
\\ \hline
%Zeile
	\textbf{Ergebnis}						&
	\begin{tabular}{l}
		Bestellung erfasst.\\
		Rechnungswesen notifiziert.\\
		Filiallager notifiziert.
	\end{tabular}
\\ \hline
%Zeile
	\textbf{Ergebnis bei Fehler}			&
	\begin{tabular}{l}
		- Fehlermeldung erscheint
	\end{tabular}
\\ \hline
%Zeile
	\textbf{Ablauf}							&
	\begin{tabular}{l}
		1. Bestellung erfassen und bestätigen.
		3. Rechnungswesen notifizieren.\\
		4. Filiallager notifizieren.\\
		5. Bestätigung.
	\end{tabular}
\\ \hline
	\textbf{Testdaten}						&
	\begin{tabular}{l}
		Testkunde.
	\end{tabular}
\\ \hline	%Untere Abgrenzung
\end{tabular}
\end{table}
% systemtests
\subsection{Systemtests}
\subsubsection{Vorbereitung}
Als generelle Vorbedingung gilt jeweils ein komplett installiertes System. Zur Installation benötigt man die Installationspakete, welche nach einem Build unter ‚build lib‘ zu finden sind. Bevor mit den Systemtests gestartet wird, sollten alle Unit- sowie Integrationstests erfolgreich durchgeführt worden sein.
% !TEX root = Dokumentation.tex
\subsubsection{Systemtest 1}
\begin{table}[H]
\begin{tabular}{ | p{0.22\textwidth} | p{0.68\textwidth} |} \hline
\rowcolor{gray!50}
%Titelzeile
	\textbf{ID}          &
	\begin{tabular}{l}
		\textbf{S1}
	\end{tabular}
	\\ \hline
%Zeile
	\textbf{Bezeichnung}			 &
	\begin{tabular}{l}
		Wareneingang erfassen.
	\end{tabular}
\\ \hline
%Zeile
	\textbf{Beschreibung}   		 &
	\begin{tabular}{l}
		Der Wareneingang wird im Filialbestellsystem erfasst und ist in der\\
Datenbank persistiert.
	\end{tabular}
\\ \hline
%Zeile
	\textbf{Akteure}              & 
	\begin{tabular}{l}
		Datentypist
	\end{tabular}
\\ \hline
%Zeile
	\textbf{Vorbedingungen}       &
	\begin{tabular}{l}
		Erfolgreich als Datentypist im System angemeldet.
	\end{tabular}		
\\ \hline
%Zeile
	\textbf{Ergebnis}             &        
	\begin{tabular}{l}
		Die erfassten Wareneingänge sind in der Datenbank persistiert.
	\end{tabular}
\\ \hline
%Zeile
	\textbf{Ergebnis bei Fehler}  &
	\begin{tabular}{l}
		- Fehlermeldung
	\end{tabular}
\\ \hline
%Zeile
	\textbf{Ablauf}				 &
	\begin{tabular}{l}
		1. Anmeldung als Datentypist\\
		2. Erfassung starten\\
		3. Persistierung\\
		4. Lagerbestand kontrollieren
	\end{tabular}
\\ \hline	%Untere Abgrenzung
\end{tabular}
\end{table}
% !TEX root = Dokumentation_PMP.tex
\subsubsection{Systemtest 1 - Automatische Nachbestellung}
\begin{table}[H]
\begin{tabular}{ | p{0.22\textwidth} | p{0.68\textwidth} |} \hline
\rowcolor{gray!50}
%Titelzeile
	\textbf{ID}          &
	\begin{tabular}{l}
		\textbf{S2}
	\end{tabular}
	\\ \hline
%Zeile
	\textbf{Bezeichnung}			 &
	\begin{tabular}{l}
		Automatische Nachbestellung auslösen.
	\end{tabular}
\\ \hline
%Zeile
	\textbf{Zugehörige Userstory}			 &
	\begin{tabular}{l}
		Nachbestellung auslösen (I8-7)
	\end{tabular}
\\ \hline
%Zeile
	\textbf{Beschreibung}   		 &
	\begin{tabular}{l}
		Das Filialbestellsystem löst automatisch eine Nachbestellung aus, wenn\\
ein definierter Lagerbestand unterschritten wird.
	\end{tabular}
\\ \hline
%Zeile
	\textbf{Akteure}              & 
	\begin{tabular}{l}
		Filialbestellsystem, Filialleiter, Filialverwalter
	\end{tabular}
\\ \hline
%Zeile
	\textbf{Vorbedingungen}       &
	\begin{tabular}{l}
		Lagerbestand unterschreitet Mindestlagerbestand.
	\end{tabular}		
\\ \hline
%Zeile
	\textbf{Ergebnis}             &        
	\begin{tabular}{l}
		Nachbestellung ausgelöst\\
		Filialleiter kann Nachbestellung einsehen\\
		Filialverwalter sieht Logeintrag.
	\end{tabular}
\\ \hline
%Zeile
	\textbf{Ergebnis bei Fehler}  &
	\begin{tabular}{l}
		- nfo an Filialleiter
	\end{tabular}
\\ \hline
%Zeile
	\textbf{Ablauf}				 &
	\begin{tabular}{l}
		1. Lagerbestand unterschreiten\\
		2. Nachbestellung auslösen\\
		3. Log-Eintrag erstellen\\
		4. Nachbestellung prüfen (Filialverwalter)
	\end{tabular}
\\ \hline	%Untere Abgrenzung
\end{tabular}
\end{table}
% !TEX root = Dokumentation.tex
\subsubsection{Systemtest 3}
\begin{table}[H]
\begin{tabular}{ | p{0.22\textwidth} | p{0.68\textwidth} |} \hline
\rowcolor{gray!50}
%Titelzeile
	\textbf{ID}          &
	\begin{tabular}{l}
		\textbf{S3}
	\end{tabular}
	\\ \hline
%Zeile
	\textbf{Bezeichnung}			 &
	\begin{tabular}{l}
		Bestellung bearbeiten.
	\end{tabular}
\\ \hline
%Zeile
	\textbf{Beschreibung}   		 &
	\begin{tabular}{l}
		Bestellungen können im System bearbeitet werden.
	\end{tabular}
\\ \hline
%Zeile
	\textbf{Akteure}              & 
	\begin{tabular}{l}
		Verkaufspersonal (Filialleiter)
	\end{tabular}
\\ \hline
%Zeile
	\textbf{Vorbedingungen}       &
	\begin{tabular}{l}
		Anmeldung als Verkaufspersonal (Filialleiter) am System.
	\end{tabular}		
\\ \hline
%Zeile
	\textbf{Ergebnis}             &        
	\begin{tabular}{l}
		Teil-Ergebnisse:\\
		 1. Bestellung kann erfasst werden.\\
		 2. Bestellung kann bearbeitet werden.\\
		 3. Bestellung kann annulliert werden.\\
		 4. Bestellung kann eingesehen werden.
		Lagerbestand aktualisiert.\\
		Rechnungswesen informiert.\\
		Bestellbestätigung ausgelöst.\\
	\end{tabular}
\\ \hline
%Zeile
	\textbf{Ergebnis bei Fehler}  &
	\begin{tabular}{l}
		- Fehlermeldung
	\end{tabular}
\\ \hline
%Zeile
	\textbf{Ablauf}				 &
	\begin{tabular}{l}
		1. Bestellung erfassen\\
		2. Bestellung ansehen.\\
		3. Bestellung bearbeiten.\\
		4. Bestellung ansehen.\\
		5. Bestellung annullieren.\\
	\end{tabular}
\\ \hline	%Untere Abgrenzung
\end{tabular}
\end{table}