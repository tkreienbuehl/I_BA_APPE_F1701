% !TEX root = Testprotokolle.tex
\subsection{Testprotokoll 27.05.2017}
Für das vorliegende Testprotokoll gilt folgendes Testkonzept in der Version x.x vom xx.xx.xxxx als Beurteilungsgrundlage. \\
Die Tests wurden durch die Person TODO ausgeführt.
\subsubsection{Testkandidaten}
Die folgenden Komponenten bzw. Maven-Projekte mit entsprechender Version werden als Testkandidaten verwendet:\\
\begin{tabular}{|l|l|}
	\hline \rowcolor{gray!50}
	\textbf{Komponente / Projekt} & \textbf{Buildversion} \\ 
	\hline 
	g13\_fbs\_client & 1.0.0 \\ 
	\hline 
	g13\_fbs\_business & 1.0.0 \\ 
	\hline 
	g13\_fbs\_data & 1.0.0 \\ 
	\hline 
	g13\_fbs\_model & 1.0.0 \\ 
	\hline 
\end{tabular} 

\subsubsection{Zusammenfassung der Ergebnisse}
TODO: Kurzes Fazit! \\
\begin{tabular}{|l|l|l|}
	\hline \rowcolor{gray!50}
	\textbf{\# Tests} & \textbf{\# Erfolgreich} & \textbf{\# Fehlgeschlagen} \\ 
	\hline 
	8 & 7 & 1 \\ 
	\hline 
\end{tabular} 

\subsubsection{Ergebnisse System- \& Integrationstests}
\begin{tabular}{|l|l|l|l|}
	\hline \rowcolor{gray!50}
	Testnr & Testbezeichnung & OK/NOK & Kommentar / Beweis \\ 
	\hline 
	I1 & Applikation 'Filialbestellsystem' kann gestartet werden & OK & alles geil! \\ 
	\hline 
	I2 & Anmeldung an Applikation funktioniert & NOK & alles scheisse! \\ 
	\hline 
	I3 & Wareneingang kann erfasst werden & OK & allles geil! \\ 
	\hline 
	I4 & Automatische Nachbestellung & OK & allles geil! Met föteli\\ 
	\hline 
	I5 & Nachbestellung einsehen & OK & allles geil! \\ 
	\hline 
	I6 & Bestellung einsehen & OK & allles geil! \\ 
	\hline
	I7 & Bestellung editieren & OK & allles geil! \\ 
	\hline  
	I8 & Bestellung erfassen & OK & allles geil! \\ 
	\hline 
	S1 & Wareneingang erfassen & OK & allles geil! \\ 
	\hline 
	S2 & Automatische Nachbestellung auslösen & OK & allles geil! \\ 
	\hline 
	S3 & Bestellung bearbeiten & OK & allles geil! \\ 
	\hline 
\end{tabular} 
\clearpage
