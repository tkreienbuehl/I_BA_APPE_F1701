% !TEX root = Dokumentation_PMP.tex
\subsubsection{Systemtest 1 - Automatische Nachbestellung}
\begin{table}[H]
\begin{tabular}{ | p{0.22\textwidth} | p{0.68\textwidth} |} \hline
\rowcolor{gray!50}
%Titelzeile
	\textbf{ID}          &
	\begin{tabular}{l}
		\textbf{S2}
	\end{tabular}
	\\ \hline
%Zeile
	\textbf{Bezeichnung}			 &
	\begin{tabular}{l}
		Automatische Nachbestellung auslösen.
	\end{tabular}
\\ \hline
%Zeile
	\textbf{Zugehörige Userstory}			 &
	\begin{tabular}{l}
		Nachbestellung auslösen (I8-7)
	\end{tabular}
\\ \hline
%Zeile
	\textbf{Beschreibung}   		 &
	\begin{tabular}{l}
		Das Filialbestellsystem löst automatisch eine Nachbestellung aus, wenn\\
ein definierter Lagerbestand unterschritten wird.
	\end{tabular}
\\ \hline
%Zeile
	\textbf{Akteure}              & 
	\begin{tabular}{l}
		Filialbestellsystem, Filialleiter, Filialverwalter
	\end{tabular}
\\ \hline
%Zeile
	\textbf{Vorbedingungen}       &
	\begin{tabular}{l}
		Lagerbestand unterschreitet Mindestlagerbestand.
	\end{tabular}		
\\ \hline
%Zeile
	\textbf{Ergebnis}             &        
	\begin{tabular}{l}
		Nachbestellung ausgelöst\\
		Filialleiter kann Nachbestellung einsehen\\
		Filialverwalter sieht Logeintrag.
	\end{tabular}
\\ \hline
%Zeile
	\textbf{Ergebnis bei Fehler}  &
	\begin{tabular}{l}
		- nfo an Filialleiter
	\end{tabular}
\\ \hline
%Zeile
	\textbf{Ablauf}				 &
	\begin{tabular}{l}
		1. Lagerbestand unterschreiten\\
		2. Nachbestellung auslösen\\
		3. Log-Eintrag erstellen\\
		4. Nachbestellung prüfen (Filialverwalter)
	\end{tabular}
\\ \hline	%Untere Abgrenzung
\end{tabular}
\end{table}