% !TEX root = Testprotokolle.tex
\subsection{Testprotokoll 01.06.2017}
Für das vorliegende Testprotokoll gilt folgendes Testkonzept in der Version x.x vom xx.xx.xxxx als Beurteilungsgrundlage. \\
Die Tests wurden durch die Person Tobias Kreienbühl ausgeführt.
\subsubsection{Testkandidaten}
Die folgenden Komponenten bzw. Maven-Projekte mit entsprechender Version werden als Testkandidaten verwendet:\\
\begin{tabular}{|l|l|}
	\hline \rowcolor{gray!50}
	\textbf{Komponente / Projekt} & \textbf{Buildversion} \\ 
	\hline 
	g13\_fbs\_client & 1.0.0 \\ 
	\hline 
	g13\_fbs\_business & 1.0.0 \\ 
	\hline 
	g13\_fbs\_data & 1.0.0 \\ 
	\hline 
	g13\_fbs\_model & 1.0.0 \\ 
	\hline 
\end{tabular} 

\subsubsection{Zusammenfassung der Ergebnisse}
Alle Testfälle konnten erfolgreich durchgeführt werden. Teils sind noch kleinere Anpassungen notwendig, welche aber nicht die Tests beeinflusst haben. \\
\begin{tabular}{|l|l|l|}
	\hline \rowcolor{gray!50}
	\textbf{\# Tests} & \textbf{\# Erfolgreich} & \textbf{\# Fehlgeschlagen} \\ 
	\hline 
	8 & 8 & 0 \\ 
	\hline 
\end{tabular} 

\subsubsection{Ergebnisse System- \& Integrationstests}
\begin{tabular}{|l|l|l|l|}
	\hline \rowcolor{gray!50}
	Testnr & Testbezeichnung & OK/NOK & Kommentar / Beweis \\ 
	\hline 
	I1 & \begin{tabular}{l}Applikation 'Filialbestellsystem' kann \\gestartet werden \end{tabular}  & OK & \begin{tabular}{l} Logindialog wird angezeigt. \\ \end{tabular} \\
	\hline 
	I2 & Anmeldung an Applikation funktioniert & OK & \begin{tabular}{l}
		Man kann sich mit den Benut-\\zer sgmuer, mmoro, tkreien-\\buehl, rwyss erfolgreich \\anmelden
	\end{tabular} \\
	\hline 
	I3 & Wareneingang kann erfasst werden & OK & \begin{tabular}{l}Artikel 1 wird mit 1000 \\
		Elemente eingelagert. \\ \end{tabular} \\ 
	\hline 
	I5 & Nachbestellung einsehen & OK & \begin{tabular}{l}Informationsgehalt lässt zu wünschen übrig. \\ \end{tabular} \\
	\hline 
	I6 & Bestellung einsehen & OK & \begin{tabular}{l}Die Testbestellung kann erfolgreich \\angezeigt werden inkl. Details und die \\
		bestellten Artikel. \\ \end{tabular} \\
	\hline
	I7 & Bestellung annullieren & OK & \begin{tabular}{l}Die Bestellung erhält den Status 'Annulliert'\\
		 und die bestellten bzw. annullierten Artikel\\
		  werden im Stock wieder freigegeben. \\ \end{tabular} \\ 
	\hline  
	I8 & Bestellung editieren & OK & \begin{tabular}{l}Der Testbestellung konnte erfolgreich ein \\
		weiterer Artikel bestellt werden und die \\
		Menge eines bereits bestellten Artikels \\
		angepasst werden. \\ \end{tabular} \\
	\hline 
	I9 & Bestellung erfassen & OK & \begin{tabular}{l}Eine neue Bestellung konnte erfolgreich\\
		 mit den entsprechenden Artikeln und dem \\
		 Kunden gespeichert werden. \\ \end{tabular} \\
	\hline 
	S1 & Wareneingang erfassen & OK & \begin{tabular}{l}Die angelieferten Artikeln konnten\\
		 erfolgreich erfasst werden und das Filiallager \\
		 wurde erfolgreich aktualisiert. \\ \end{tabular} \\
	\hline 
	S2 & Automatische Nachbestellung auslösen & OK & \begin{tabular}{l}Eine Nachbestellung wird erfolgreich bei \\
		Unterschreitung von Menge 2 ausgelöst \end{tabular} \\ 
	\hline 
	S3 & Bestellung bearbeiten & OK & \begin{tabular}{l}Bei Bestellung von mehr als im Filiallager\\
		 verfügbarer Menge wird neben dem ge-\\
		 änderten Filiallagerbestand (negativ) einen\\
		  auch eine Nachbestellung am Zentrallager\\
		   ausgelöst. \\ \end{tabular} \\
	\hline 
\end{tabular} 
\clearpage
