% !TEX root = Dokumentation_PMP.tex
\subsubsection{Integrationstest 4 - Auto Nachbestellung}
\begin{table}[H]
\begin{tabular}{ | p{0.22\textwidth} | p{0.68\textwidth} |} \hline
\rowcolor{gray!50}
%Titelzeile
	\textbf{ID}						&
	\begin{tabular}{l}
		\textbf{I4}
	\end{tabular}
	\\ \hline
%Zeile
	\textbf{Bezeichnung}			&
	\begin{tabular}{l}
		Automatische Nachbestellung.
	\end{tabular}
\\ \hline
%Zeile
	\textbf{Beschreibung}			&
	\begin{tabular}{l}
		Bei Unterschreiten eines Mindestbestandes wird eine automatische\\
Bestellung ausgelöst.
	\end{tabular}
\\ \hline
%Zeile
	\textbf{Akteure}				& 
	\begin{tabular}{l}
		Filiallager
	\end{tabular}
\\ \hline
%Zeile
	\textbf{Vorbedingungen}			&
	\begin{tabular}{l}
		Lagerbestand von Artikel unterschreitet Mindestbestand.
	\end{tabular}		
\\ \hline
%Zeile
	\textbf{Ergebnis}				&        
	\begin{tabular}{l}
		Nachbestellung ausgelöst.
	\end{tabular}
\\ \hline
%Zeile
	\textbf{Ergebnis bei Fehler}	&
	\begin{tabular}{l}
		- Fehlermeldung erscheint\\
		- Meldung an Filialleiter
	\end{tabular}
\\ \hline
%Zeile
	\textbf{Ablauf}					&
	\begin{tabular}{l}
		1. Bestellung aufgegeben\\
		2. Lagerbestand unterschreitet Mindestbestand\\
		3. Bestellung ausgelöst
	\end{tabular}
\\ \hline
	\textbf{Testdaten}				&
	\begin{tabular}{l}
		Artikel mit Lagermenge an Mindestgrenze.
	\end{tabular}
\\ \hline	%Untere Abgrenzung
\end{tabular}
\end{table}