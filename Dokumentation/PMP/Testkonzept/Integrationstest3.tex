% !TEX root = Dokumentation.tex
\subsubsection{Integrationstest 3 - Erfassung Wareneingang}
\begin{table}[H]
\begin{tabular}{ | p{0.22\textwidth} | p{0.68\textwidth} |} \hline
\rowcolor{gray!50}
%Titelzeile
	\textbf{ID}						&
	\begin{tabular}{l}
		\textbf{I3}
	\end{tabular}
	\\ \hline
%Zeile
	\textbf{Bezeichnung}			&
	\begin{tabular}{l}
		Wareneingang kann erfasst werden.
	\end{tabular}
\\ \hline
%Zeile
	\textbf{Beschreibung}			&
	\begin{tabular}{l}
		Die Erfassung von Lieferungen ans Filiallager kann erfasst werden.
	\end{tabular}
\\ \hline
%Zeile
	\textbf{Akteure}				& 
	\begin{tabular}{l}
		Datentypist
	\end{tabular}
\\ \hline
%Zeile
	\textbf{Vorbedingungen}			&
	\begin{tabular}{l}
		Applikation erfolgreich gestartet\\
Anmeldung erfolgreich\\
Benutzergruppe Datentypist zugewiesen
	\end{tabular}		
\\ \hline
%Zeile
	\textbf{Ergebnis}				&        
	\begin{tabular}{l}
		Bestätigung der erfassten Lieferung.
	\end{tabular}
\\ \hline
%Zeile
	\textbf{Ergebnis bei Fehler}	&
	\begin{tabular}{l}
		- Fehlermeldung erscheint
	\end{tabular}
\\ \hline
%Zeile
	\textbf{Ablauf}					&
	\begin{tabular}{l}
		1. Formular ausfüllen\\
		2. Formular absenden\\
		3. Bestätigung erhalten
	\end{tabular}
\\ \hline
	\textbf{Testdaten}				&
	\begin{tabular}{l}
		Test-Lieferung.
	\end{tabular}
\\ \hline	%Untere Abgrenzung
\end{tabular}
\end{table}