% !TEX root = Dokumentation_PMP.tex
\subsubsection{Integrationstest 1 - Applikationsstart}
\begin{table}[H]
\begin{tabular}{ | p{0.22\textwidth} | p{0.68\textwidth} |} \hline
\rowcolor{gray!50}
%Titelzeile
	\textbf{ID}          &
	\begin{tabular}{l}
		\textbf{I1}
	\end{tabular}
	\\ \hline
%Zeile
	\textbf{Bezeichnung}			 &
	\begin{tabular}{l}
		Applikation \glqq{}Filialbestelsystem\grqq{} kann gestartet werden.
	\end{tabular}
\\ \hline
%Zeile
	\textbf{Beschreibung}   		 &
	\begin{tabular}{l}
		Die Applikation kann gestartet werden.
	\end{tabular}
\\ \hline
%Zeile
	\textbf{Akteure}              & 
	\begin{tabular}{l}
		Nutzer
	\end{tabular}
\\ \hline
%Zeile
	\textbf{Vorbedingungen}       &
	\begin{tabular}{l}
		keine
	\end{tabular}		
\\ \hline
%Zeile
	\textbf{Ergebnis}             &        
	\begin{tabular}{l}
		Integrationstest werden fehlerfrei abgeschlossen. \\
		Die Applikation ist erfolgreich gestartet. \\
		Die Anmelde-Maske wird gestartet. \\
	\end{tabular}
\\ \hline
%Zeile
	\textbf{Ergebnis bei Fehler}  &
	\begin{tabular}{l}
		- Exception\\
		- Unit Test failed
	\end{tabular}
\\ \hline
%Zeile
	\textbf{Ablauf}				 &
	\begin{tabular}{l}
		1. Applikation starten\\
		2. Ergebnis prüfen
	\end{tabular}
\\ \hline
	\textbf{Testdaten}            &
	\begin{tabular}{l}
		Keine benötigt.
	\end{tabular}
\\ \hline	%Untere Abgrenzung
\end{tabular}
\end{table}