% !TEX root = Dokumentation.tex
\subsubsection{Systemtest 1}
\begin{table}[H]
\begin{tabular}{ | p{0.22\textwidth} | p{0.68\textwidth} |} \hline
\rowcolor{gray!50}
%Titelzeile
	\textbf{ID}          &
	\begin{tabular}{l}
		\textbf{S1}
	\end{tabular}
	\\ \hline
%Zeile
	\textbf{Bezeichnung}			 &
	\begin{tabular}{l}
		Wareneingang erfassen.
	\end{tabular}
\\ \hline
%Zeile
	\textbf{Beschreibung}   		 &
	\begin{tabular}{l}
		Der Wareneingang wird im Filialbestellsystem erfasst und ist in der\\
Datenbank persistiert.
	\end{tabular}
\\ \hline
%Zeile
	\textbf{Akteure}              & 
	\begin{tabular}{l}
		Datentypist
	\end{tabular}
\\ \hline
%Zeile
	\textbf{Vorbedingungen}       &
	\begin{tabular}{l}
		Erfolgreich als Datentypist im System angemeldet.
	\end{tabular}		
\\ \hline
%Zeile
	\textbf{Ergebnis}             &        
	\begin{tabular}{l}
		Die erfassten Wareneingänge sind in der Datenbank persistiert.
	\end{tabular}
\\ \hline
%Zeile
	\textbf{Ergebnis bei Fehler}  &
	\begin{tabular}{l}
		- Fehlermeldung
	\end{tabular}
\\ \hline
%Zeile
	\textbf{Ablauf}				 &
	\begin{tabular}{l}
		1. Anmeldung als Datentypist\\
		2. Erfassung starten\\
		3. Persistierung\\
		4. Lagerbestand kontrollieren
	\end{tabular}
\\ \hline	%Untere Abgrenzung
\end{tabular}
\end{table}