% !TEX root = Dokumentation.tex
\subsubsection{Integrationstest 2}
\begin{table}[H]
\begin{tabular}{ | p{0.22\textwidth} | p{0.68\textwidth} |} \hline
\rowcolor{gray!50}
%Titelzeile
	\textbf{ID}          &
	\begin{tabular}{l}
		\textbf{I2}
	\end{tabular}
	\\ \hline
%Zeile
	\textbf{Bezeichnung}			 &
	\begin{tabular}{l}
		Anmeldung an Applikation funktioniert.
	\end{tabular}
\\ \hline
%Zeile
	\textbf{Beschreibung}   		 &
	\begin{tabular}{l}
		Die Anmeldung mit Username und Passwort funktioniert\\
		an der Applikation.
	\end{tabular}
\\ \hline
%Zeile
	\textbf{Akteure}              & 
	\begin{tabular}{l}
		Benutzer
	\end{tabular}
\\ \hline
%Zeile
	\textbf{Vorbedingungen}       &
	\begin{tabular}{l}
		Applikation erfolgreich gestartet
	\end{tabular}		
\\ \hline
%Zeile
	\textbf{Ergebnis}             &        
	\begin{tabular}{l}
		Anmeldung erfolgreich\\
Benutzeransicht nach Benutzergruppe erscheint
	\end{tabular}
\\ \hline
%Zeile
	\textbf{Ergebnis bei Fehler}  &
	\begin{tabular}{l}
		- Fehlermeldung erscheint\\
	\end{tabular}
\\ \hline
%Zeile
	\textbf{Ablauf}				 &
	\begin{tabular}{l}
		1. Username eingeben\\
		2. Passwort eingeben\\
		3. Anmeldung bestätigen
	\end{tabular}
\\ \hline
	\textbf{Testdaten}            &
	\begin{tabular}{l}
		Username, Passwort und Gruppenzugehörigkeit.
	\end{tabular}
\\ \hline	%Untere Abgrenzung
\end{tabular}
\end{table}