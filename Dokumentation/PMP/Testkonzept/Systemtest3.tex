% !TEX root = Dokumentation_PMP.tex
\subsubsection{Systemtest 3 - Bearbeitung Bestellung}
\begin{table}[H]
\begin{tabular}{ | p{0.22\textwidth} | p{0.68\textwidth} |} \hline
\rowcolor{gray!50}
%Titelzeile
	\textbf{ID}          &
	\begin{tabular}{l}
		\textbf{S3}
	\end{tabular}
	\\ \hline
%Zeile
	\textbf{Bezeichnung}			 &
	\begin{tabular}{l}
		Bestellung bearbeiten.
	\end{tabular}
\\ \hline
%Zeile
	\textbf{Beschreibung}   		 &
	\begin{tabular}{l}
		Bestellungen können im System bearbeitet werden.
	\end{tabular}
\\ \hline
%Zeile
	\textbf{Akteure}              & 
	\begin{tabular}{l}
		Verkaufspersonal (Filialleiter)
	\end{tabular}
\\ \hline
%Zeile
	\textbf{Vorbedingungen}       &
	\begin{tabular}{l}
		Anmeldung als Verkaufspersonal (Filialleiter) am System.
	\end{tabular}		
\\ \hline
%Zeile
	\textbf{Ergebnis}             &        
	\begin{tabular}{l}
		Teil-Ergebnisse:\\
		 1. Bestellung kann erfasst werden.\\
		 2. Bestellung kann bearbeitet werden.\\
		 3. Bestellung kann annulliert werden.\\
		 4. Bestellung kann eingesehen werden.
		Lagerbestand aktualisiert.\\
		Rechnungswesen informiert.\\
		Bestellbestätigung ausgelöst.\\
	\end{tabular}
\\ \hline
%Zeile
	\textbf{Ergebnis bei Fehler}  &
	\begin{tabular}{l}
		- Fehlermeldung
	\end{tabular}
\\ \hline
%Zeile
	\textbf{Ablauf}				 &
	\begin{tabular}{l}
		1. Bestellung erfassen\\
		2. Bestellung ansehen.\\
		3. Bestellung bearbeiten.\\
		4. Bestellung ansehen.\\
		5. Bestellung annullieren.\\
	\end{tabular}
\\ \hline	%Untere Abgrenzung
\end{tabular}
\end{table}