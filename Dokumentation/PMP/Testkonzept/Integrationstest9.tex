% !TEX root = Dokumentation_PMP.tex
\subsubsection{Integrationstest 9 - Bestellung erfassen}
\begin{table}[H]
\begin{tabular}{ | p{0.22\textwidth} | p{0.68\textwidth} |} \hline
\rowcolor{gray!50}
%Titelzeile
	\textbf{ID}								&
	\begin{tabular}{l}
		\textbf{I7}
	\end{tabular}
	\\ \hline
%Zeile
	\textbf{Bezeichnung}					&
	\begin{tabular}{l}
		Bestellung erfassen.
	\end{tabular}
\\ \hline
%Zeile
	\textbf{Beschreibung}					&
	\begin{tabular}{l}
		Eine Bestellung kann erfasst
werden.
	\end{tabular}
\\ \hline
%Zeile
	\textbf{Akteure}						&
	\begin{tabular}{l}
		Filialleiter, Verkaufspersonal.
	\end{tabular}
\\ \hline
%Zeile
	\textbf{Vorbedingungen}					&
	\begin{tabular}{l}
		Erfolgreich an System angemeldet.\\
		Benutzergruppe zugewiesen.\\
		Für Testkunde keine offenen Mahnungen vorhanden.
	\end{tabular}		
\\ \hline
%Zeile
	\textbf{Ergebnis}						&
	\begin{tabular}{l}
		Bestellung erfasst.\\
		Rechnungswesen notifiziert.\\
		Filiallager notifiziert.
	\end{tabular}
\\ \hline
%Zeile
	\textbf{Ergebnis bei Fehler}			&
	\begin{tabular}{l}
		- Fehlermeldung erscheint
	\end{tabular}
\\ \hline
%Zeile
	\textbf{Ablauf}							&
	\begin{tabular}{l}
		1. Bestellung erfassen und bestätigen.
		3. Rechnungswesen notifizieren.\\
		4. Filiallager notifizieren.\\
		5. Bestätigung.
	\end{tabular}
\\ \hline
	\textbf{Testdaten}						&
	\begin{tabular}{l}
		Testkunde.
	\end{tabular}
\\ \hline	%Untere Abgrenzung
\end{tabular}
\end{table}